\section{Conclusion}
Taking advantage of the structure inherent in tensor train decompositions, we propose and
analyze the Orthogonal Tensor Train Decomposition, yielding direct benefits in both parameter
efficiency and computation time. This is an important step
% which turns out be important in
in instantiating recurrent or sequential models for a set of longitudinal 3D brain images,
either in the context of generating new images in the sequence or for classification. 
Using a mapping from Euclidean space, we construct a neural network
variable that can efficiently be learned through existing deep learning optimization frameworks.
Our results yield promising
developments in applying 
deep learning methods for analyzing sequential 3D medical imaging data, and
we show that our method can perform favorably in reconstruction and
prediction tasks with such image volumes.
%, with minimal loss in ability to quantify uncertainty.
While a focus here was brain imaging,
we anticipate numerous applications in other medical imaging settings.
%{\color{red}and our code will be made
%available to facilitate additional research. }
Code is available at \url{https://github.com/ronakrm/OTT}.
