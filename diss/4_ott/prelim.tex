\section{Preliminaries}\label{sec:prelim}
\subsection{Tensor Decompositions and Tensor Trains. }
Let $\cX \in \RR^{n_1 \times \cdots \times n_d}$ be a $d$-dimensional array, or tensor, with each mode having length $n_i$. To store a full rank tensor, $n^d$ storage would be required. A number of tensor factorizations have been developed to reduce this storage cost. The CANDECOMP/PARAFAC (CP) \cite{harshman1970foundations,carroll1970analysis} decomposition reduces the storage to $O(dnr)$, but finding the exact CP-rank $r$ is NP-hard. Hierarchical tensor methods have also proven to be effective in tensor compression \cite{cohen2016expressive, cohen2016convolutional}.

A more recent decomposition, the \textit{Tensor Train} decomposition (TT) \cite{oseledets2011tensor}, defines an element of the tensor as
\begin{align}
\cX(x_1,\ldots,x_d) = A_1(x_1)\cdots A_d(x_d)
\end{align}
where $x_i \in \left\{1,\ldots, n_i\right\}$, and $A_i(x_i) \in \RR^{r_{i-1} \times r_{i}}$ for each $i \in \{1,\ldots,d\}$ are called the \textit{cores} of the tensor train, with $r_0 = r_d = 1$. Equivalently, the full tensor is written as:
\begin{align}\label{eq:fullTT}
    \cX &= \sum_{k_0 = 1}^{r_0} \cdots \sum_{k_d = 1}^{r_d} A_1(k_0,:,k_1) \otimes \cdots \otimes A_d(k_{d-1},:,k_d) 
\end{align}
where $A_i \in \RR^{r_{i-1}\times n_i \times r_i} $. This format requires $O(dnr^2)$ storage, but has two major advantages over the CP format. First, finding the TT-rank (the smallest set of $r_i$'s that satisfy the decomposition with equality) of any arbitrary tensor is tractable, and as such all tensors can be efficiently rewritten in the TT format. Second, projecting arbitrary tensors onto the TT format of a fixed rank requires only a set of QR and singular value decompositions \cite{oseledets2011tensor}. This projection, \textit{TT-rounding}, additionally allows for a given TT tensor of some rank to be projected onto the space of TTs with lower rank, and requires $O(dr^3)$ computational complexity. Separately, specific tensor train constructions have recently been identified as forms of general recurrent networks \cite{khrulkov2018generalized}.

We denote a tensor operator $\tG$ as a grouping of tensor modes into an ``input" and ``output" list, such that $\tG \in \RR^{(n_1^{in},\times\ldots\times,n_d^{in}) \times (n_1^{out},\times\ldots\times,n_d^{out})}$. This operator $\tG$ can be seen as the TT representation of a matrix $W \in \RR^{(n_1^i\cdots n_d^i) \times (n_1^o\cdots n_d^o)}$. In \cite{novikov2015tensorizing}, authors use this formulation to directly compress the weight layers in neural networks. Cores in the operator are indexed by both an input and output index, i.e., $A_i(x_i,y_i) \in \RR^{r_{i-1}\times r_i}$, where $x_i \in [1,\ldots,n_i^{in}], y_i \in [1,\ldots,n_i^{out}]$.

Common operations on tensor trains require \textit{matricizing} the cores of the TT format. Here, we define the left matricization of core $A_i(x_i)$ as $A_i^L \in \RR^{r_{i-1}n_i \times r_i} $ and the right matricization similarly.

\subsection{Differential Geometry of Tensor Trains}
Tensor trains with fixed TT-ranks form a Riemannian submanifold of $\RR^{n_1 \times \cdots \times n_d}$ \cite{lubich2015time, holtz2012manifolds}:
\begin{align}\label{eq:riem}
    \cM_r := \{ \cX \in \RR^{n_1 \times \cdots \times n_d} \text{ with TT-ranks } r_0,\ldots,r_d\} 
\end{align}
A Riemannian manifold is a differentiable manifold with a smoothly varying inner product. The tangent space is a vector space defined at a specific point on the manifold, consisting of all possible tangent vectors of all possible curves along the manifold passing through that point. The tangent bundle, the disjoint union of all tangent spaces for all points on the manifold, is canonically equipped with a \textit{projection map}: $\Pi: T\cM \rightarrow \cM$. The Exponential Map defines a local map from the tangent space at a specific point on the manifold $\Exp(x,\cdot): T_x\cM \rightarrow \cM$. With these definitions, optimizing a function with respect to a Riemannian manifold-valued variable amounts to computing a free derivative in the ambient space, projecting the gradient to the tangent space of the current iterate, and using the (retraction) exponential map to compute the next iterate on the manifold.
The authors in \cite{novikov2016exponential} use this procedure to more effectively learn a model of all exponentially many interactions in a linear model.

Orthogonal matrices of fixed size and rank also form a manifold, the (compact) Stiefel Manifold: $ \ST(p,n)=\left\{Y \in \RR^{n\times p} | Y^TY=I_p,\: p \leq n\right\}$.
An arbitrary $X \in \RR^{n \times p}$ matrix can be projected onto the Stiefel manifold $\ST(p,n)$ using $X \mapsto UV^T$ where $X=U\Sigma V^T$ is the (thin) singular value decomposition of $X$. 
