\usepackage[framemethod=TikZ]{mdframed}

% \newtheorem{theorem}{Theorem}
% \newtheorem{lemma}{Lemma}
% \newtheorem{proposition}{Proposition}
% \newtheorem{definition}{Definition}
% \newtheorem{corollary}{Corollary}
% \newtheorem{remark}{Remark}
% \newtheorem{assumption}{Assumption}

\SetKwComment{Comment}{$\triangleright$\ }{}

\DeclareMathOperator*{\argmax}{arg\,max}
\DeclareMathOperator*{\argmin}{arg\,min}
\DeclareMathOperator*{\triu}{triu}


\newcommand{\norm}[1]{\left\|#1\right\|}
\newcommand{\normlzero}{L^0}
\newcommand{\normlone}{L^1}
\newcommand{\normltwo}{L^2}
\newcommand{\normlp}{L^p}
\newcommand{\normmax}{L^\infty}

\newcommand\myas{\stackrel{\mathclap{\normalfont\mbox{a.s.}}}{=}}

\def\CC{\mathbb{C}}
\def\EE{\mathbb{E}}
\def\II{\mathbb{I}}
\def\NN{\mathbb{N}}
\def\PP{\mathbb{P}}
\def\QQ{\mathbb{Q}}
\def\RR{\mathbb{R}}
\def\ZZ{\mathbb{Z}}

\def\Exp{\text{exp}}
\def\ST{\text{St}}


\def\cA{\mathcal{A}}
\def\cC{\mathcal{C}}
\def\cD{\mathcal{D}}

\def\cG{\mathcal{G}}
\def\tG{\mathcal{G}}

\def\cH{\mathcal{H}}
\def\cI{\mathcal{I}}
\def\cJ{\mathcal{J}}
\def\cK{\mathcal{K}}
\def\cL{\mathcal{L}}

\def\cM{\mathcal{M}}
\def\Mc{\mathcal{M}} % back compat

\def\cP{\mathcal{P}}
\def\cR{\mathcal{R}}
\def\cS{\mathcal{S}}
\def\cT{\mathcal{T}}
\def\cU{\mathcal{U}}
\def\cV{\mathcal{V}}
\def\cW{\mathcal{W}}

\def\cX{\mathcal{X}}
\def\tX{\mathcal{X}}

\def\cY{\mathcal{Y}}
\def\cZ{\mathcal{Z}}

%\newcommand{\BV}{\bm{V}}
\def\BV{\bm{V}}
%\newcommand{\Bbeta}{\bm{\beta}}
\def\Bbeta{\bm{\beta}}
\newcommand{\defeq}{\mathrel{\mathop:}=}

\DeclareMathOperator*{\SPD}{SPD}
\DeclareMathOperator*{\VAR}{VAR}
\DeclareMathOperator*{\EXP}{Exp}
\DeclareMathOperator*{\LOG}{Log}

\DeclareMathOperator{\vvec}{vec}
\DeclareMathOperator{\tr}{tr}
\DeclareMathOperator{\diag}{diag}
\DeclareMathOperator{\rank}{rank}

%%%  Independence symbols
\makeatletter
\newcommand*{\indep}{%
  \mathbin{%
    \mathpalette{\@indep}{}%
  }%
}
\newcommand*{\nindep}{%
  \mathbin{%                   % The final symbol is a binary math operator
    \mathpalette{\@indep}{\not}% \mathpalette helps for the adaptation
                               % of the symbol to the different math styles.
  }%
}
\newcommand*{\@indep}[2]{%
  % #1: math style
  % #2: empty or \not
  \sbox0{$#1\perp\m@th$}%        box 0 contains \perp symbol
  \sbox2{$#1=$}%                 box 2 for the height of =
  \sbox4{$#1\vcenter{}$}%        box 4 for the height of the math axis
  \rlap{\copy0}%                 first \perp
  \dimen@=\dimexpr\ht2-\ht4-.2pt\relax
      % The equals symbol is centered around the math axis.
      % The following equations are used to calculate the
      % right shift of the second \perp:
      % [1] ht(equals) - ht(math_axis) = line_width + 0.5 gap
      % [2] right_shift(second_perp) = line_width + gap
      % The line width is approximated by the default line width of 0.4pt
  \kern\dimen@
  {#2}%
      % {\not} in case of \nindep;
      % the braces convert the relational symbol \not to an ordinary
      % math object without additional horizontal spacing.
  \kern\dimen@
  \copy0 %                       second \perp
} 
\makeatother



\mdfdefinestyle{MyFrame}{%
    linecolor=blue!50!white,
    outerlinewidth=1pt,
    roundcorner=5pt,
    innertopmargin=0.5\baselineskip,
    innerbottommargin=0.5\baselineskip,
    innerrightmargin=10pt,
    innerleftmargin=10pt,
    backgroundcolor=blue!10!white}