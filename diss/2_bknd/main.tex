\chapter{Background}\label{chap:bknd}
Here we will briefly describe some background concepts and ideas that will aid and facilitate discussion of subset identification in later chapters. 
Following some general notation,
we will begin with an overview
of classical probability and statistics,
and a focus on the hypothesis testing
schemes built upon in the third chapter.
This is followed by the basics of Riemannian differential geometry
and the geometry of tensor objects,
the key objects that enable the 
tests and efficiencies described in both 
Chapters \ref{chap:covtraj} and \ref{chap:ott}.
Next we will provide
an overview of optimization tools
and methods particularly suited
and developed for neural network methods,
alongside typical forms and objectives 
in the machine learning literature,
used in Chapter \ref{chap:lcodec}.
We conclude this chapter
with a discussion on optimal transport
methods in deep learning, important
for our discussion of novel optimization schemes
in Chapter \ref{chap:demd}.

\section{General Notations}
The following notations will be standard throughout the sequel, with any particular chapter overloads or redefinitions explicitly mentioned.
\begin{itemize}
\item Calligraphic capital letters such as $\cD, \cM, \cX$ will typically represent spaces.

\item Lowercase letters such as $i,x,n$ will represent vectors as well as scalars (indices, sizes, and dimensions), and subsequently a vector $v \in \RR^d$ would be a $d$-dimensional vector over the reals.

\item Capital letters such as $X,H,T$ will refer to matrices and higher order tensors, as well as sets (including random variables) based on context.

\item We use $\RR^d$ to represent $d$-dimensional vector space over the reals.

\item For learning settings, the training dataset will typically be defined by $S$, with samples $x_i \in S$ in the general setting, and $(x_i,y_i)\in S$ in the supervised setting. Other datasets $T,U$ may also be used and defined in context.

\item Subscripts will denote indexing into sets or vectors, or estimation (in the case of integrals and expectations) over the specified subscript. Superscripts will be used as additional decorators when needed.

\end{itemize}

\section{Probability and Independence}

A random variable $X$ is a mapping from each element in an outcome space $x \in X$ to a real number $\RR$. This mapping is generally written as $p(x):=\PP(X=x), x \in X$ to denote the distribution over possible realizations $x$ of $X$. $X$ may also denote a \textit{set} of random variables $X:=\{x_i\}_{i=1}^n$, where a draw from the distribution over $X$ results in a vector $x$ of dimension $n$.
The \textit{expectation} of a random variable $\EE[X]$ is the ``weighted average'' over all outcomes. For discrete variables (which will be our main focus with expectations), 
\begin{align}
\EE[X] = \sum_{x\in X} x\cdot p(x)
\end{align}
The properties of distributions below extend to measures of expectations among multiple variables, with some caveats mentioned in specific chapters as needed.

Over a set of random variables $X$ and $Y$, the distribution $p(x,y):= \PP(X=x,Y=y)$ is the \textit{joint} distribution both variables $(X,Y)$. The \textit{marginal} distribution for $x$ is computed by ``summing out'' $y$, i.e.,
\begin{align}
p(x) = \PP(X=x) &= \sum_y \PP(X=x,Y=y) = \sum_y p(x,y) \\
&= \int p(x,y) dy
\end{align}
We say that the random variables $X$ and $Y$ are \textit{independent} if their joint distribution is equal to the product of their marginals: $p(x,y) = p(x)p(y)$.
Intuitively, a draw from one distribution has no impact on the draw from the other. In the case where the variables are \textit{dependent}, the distributions are linked. The dependent draw is defined by the \textit{conditional} distribution $p(y|x)$. Then, a joint distribution can factor as an iterative draw first from $p(x)$ followed by from $p(y|x)$: $p(x,y) = p(x)p(y|x)$. Importantly, this dependence is commutative: it is also true that $p(x,y) = p(y)p(x|y)$. When $p(y|x) = p(y)$, we say that $Y$ is independent of $X$. This commutativity leads to the formulation of Bayes' Rule:
\begin{align}
    p(y|x) = \frac{p(x|y)p(y)}{p(x)}
\end{align}
This form is used throughout statistics and machine learning. In most settings, we are interested in estimating some parameters $\theta$, or distribution over parameters $p(\theta)$, given a distribution over data $p(x)$. Applying Bayes' Rule:
\begin{align}
    p(\theta|x) = \frac{p(x|\theta)p(\theta)}{p(x)}
\end{align}
Where $p(x|\theta)$ is the \textit{likelihood} of observing the data $x$ given some parameter set, $p(\theta)$ is the \textit{prior} assumption on the base distribution over parameters, $p(x)$ is the marginal over all data (treated as a normalizing constant), and $p(\theta|x)$ is the \textit{posterior} distribution over the parameters given some data.
Alternatively, we wish to update our estimation of the parameters $\theta$ from $p(\theta)$ to $p(\theta|x)$ given some observations $x$ that change our beliefs about the parameter space.
Maximum a posteriori (MAP) estimation attempts to find the parameters that maximize this posterior, using Bayes' rule for tractable computation and estimation.
\begin{align}\label{eq:map}
	\max_\theta p(\theta|x) = \max_\theta \frac{p(x|\theta)p(\theta)}{p(x)} =  \max_\theta p(x|\theta)p(\theta)
\end{align}
If we have some samples $\{x_i\}_{i=1}^n$ from the data distribution that comprise a dataset,
then we have the following equivalent log-transformed model:
\begin{align}
	\max_\theta \sum_{i=1}^n \log p(x_i|\theta) + \log p(\theta)
\end{align}
This is the canonical form used in most of machine learning: generally associated with a ``loss" or recovery term to the likelihood, and a regularization to the prior.\todo{grammar}

\subsection{Conditional Independence} 
With three or more variables, independence relations among variables may be determined by which variables are being conditioned upon. 
\begin{definition}[Conditional Independence]\label{def:condindep}
For three random variables $X,Y,Z$, we say that 
$Y$ is \textit{conditionally independent} of $X$ given $Z$, written as $Y \indep X | Z$, if
\begin{align}
    p(y,x|z) = p(y|z)p(x|z).
\end{align}    
\end{definition}
Intuitively, once you know $Z$, $X$ provides no additional information in predicting $X$. This can also be written as $p(y|x,z)=p(y|z)$.



\paragraph{Graphs.}
\todo{fig: pgm,graph, precision}
With a larger number of variables, independence relations can be difficult to define with this notation. \textit{Graphs} are commonly used in place.
Graphs $G \in \cG$ are defined by their vertex and edge sets $G:=(E,V)$.
The vertices $V$ correspond to some random variables $X,Y,Z,\ldots$, or $X_1,\ldots,X_n$. 
An \textit{undirected} graph is one in which the existence of an edge $e_{ij}$ implies the existence of edge $e_{ji}$.
Within a \textit{probabilistic graphical model}, an edge $e_{ij}$ implies a \textit{conditional dependence}. The omission of a particular edge thus has an explicit meaning:
\begin{align}
    e_{ij} \not\in E \iff X_i \indep X_j | X_{V\setminus(i,j)}
\end{align}
Where $V\setminus(i,j)$ is the rest of the variables in the set and graph. A graph with these independence relations is also referred to as a Markov graph: pairwise conditional independence relations imply \textit{global} conditional independence relations: for any sets of variables $A,B,C$, if $C$ ``separates'' $A$ and $B$, 
While directed graphs are used and are of independent interest, in this thesis we focus on undirected graphs.

\paragraph{Multivariate Gaussians.}
Extending the typical univariate Gaussian distribution $p(x) = \frac{1}{\sqrt{2\pi\sigma^2}}e^{-(x-\mu)^2/2\sigma^2}$ defined by mean and variance parameters $\mu,\sigma^2$, we have the following density function for multivariate Gaussian distributions over $n$ variables, with realizations as the vector $\mathbf{x}:=[x_1,\ldots,x_n]^\top$:
\begin{align}\label{eq:mvgauss}
    p(x_1,\ldots,x_n) = \frac{1}{\sqrt{(2\pi)^n|\Sigma|}} \EXP\left(\frac{1}{2}(\mathbf{x} - \mathbf{\mu})^\top \Sigma^{-1} (\mathbf{x} - \mathbf{\mu})\right)
\end{align}
\todo{exp def}
Where $\mathbf{\mu}$ is the vector of means for each individual variable, and $\Sigma$ is the $n\times n$ \textit{covariance} matrix describing the second-order interactions among the variables. 
The tractability of both computing the density and estimating it in typical likelihood estimation frameworks makes the multivariate Gaussian a particular attractive prior used in many applications,
including estimating independence.
Specifically, 
\begin{theorem}\label{thm:mvnindep}
    Let $X$ be governed by a multivariate Gaussian as defined in \eqref{eq:mvgauss}. Then it holds that
    \begin{align}
        \Sigma_{ij} = 0 \iff X_i \indep X_j
    \end{align}
\end{theorem}
Complete independence among variables is often not possible and in most settings not valuable. However, a more powerful result for conditional independence exists.
\begin{theorem}\citep{lauritzen1996graphical}\label{thm:mvncondindep}
    Let $X$ be governed by a multivariate Gaussian as defined in \eqref{eq:mvgauss}, and $\Omega:=\Sigma^{-1}$ be the precision matrix. Then it holds that
    \begin{align}
        \Omega_{ij} = 0 \iff X_i \indep X_j | X_{\setminus(i,j)}
    \end{align}
\end{theorem}
This immediately yields that the precision matrix encodes the edges of the probabilistic graphical model over the variables. Estimating these dependencies is thus equivalent to estimation of the precision matrix.
Further, if the precision matrix is sparse, we can  
derive dependencies between features when the data are high-dimensional and/or the number of measurements are small. 


\subsection{Estimating Parameters and Measures Over and Among Distributions}
Multivariate data analysis exploiting the conditional independence structure between features or covariates using 
undirected graphical models is now standard within any data analysis toolbox. 
When samples are drawn from a particular family of distributions, a number of methods exist to estimate the parameters that fit that data best. 
The estimation of a graphical model
has been extensively studied
and a rich literature is available describing 
its statistical and algorithmic properties \citep{koller2009probabilistic,jordan1998learning}. 
MAP estimation and maximum likelihood estimation as described above are typically used,
but in many cases their standard forms do not yield
parameters that reveal independence, i.e., it is often impossible for standard
methods to result in a parameter estimate of zero.

To determine the relationship among two random variables, measures such as the Pearson correlation coefficient, and Spearman and Kendall rank coefficients \citep{myers2013research} are frequently used, with values close to zero suggesting a low importance, and a value close to 1 suggesting perfect dependence. However,
these measures are only applicable with specific assumptions about the possible dependencies between the variables: Pearson coefficients can only identify linear dependencies, and rank coefficients typically fail with non-monotonic dependencies (e.g., periodic functions). These methods can be computed pairwise among all variables among a set of variables with size greater than 2, however, to estimate conditional dependencies typically requires the estimation of \textit{partial} coefficient measures.
Additionally, while results similar to Theorem~\ref{thm:mvncondindep} exist under specific assumptions ,
practical estimation does not often yield exact zeros,
and estimating all partial coefficients separately can be computationally heavy.

With some assumptions, \textit{sparse} recoveries are possible while estimating ALL conditional independencies. Following the results from multivariate Gaussians above,
a \textit{penalized} version of a maximum likelihood estimate can be recovered through the following \textit{graphical lasso} \citep{friedman2008sparse} formulation:
\begin{align}\label{eq:glasso}
    \hat{\Omega} = \min_{\Omega \succeq 0} \tr(S\Omega) + \log|\Omega| + \lambda||\Omega||_1
\end{align}
\todo{semidefinite def}
With $S$ being the sample covariance of the data samples, and $\lambda$ a penalization weight. This $\ell_1$ penalization has been shown to be easy to compute, and a number of alternative versions have been proposed with varying theoretical properties and recovery guarantees~\citep{cai2011constrained,yuan2010high}.
Particularly interesting are extensions to \textit{nonparanormal} distributions, which allow for a much larger set of graphs to be estimated over other distributions via a rank covariance matrix, generalizing the pairwise rank and distance coefficients above \citep{liu2009nonparanormal,xue2012regularized}.

The distributional assumptions needed for all of these methods, however, are often completely unknown in practice, or the data represent highly complex densities and functions that cannot be represented by simple exponential families or rank-based measures.
Newer measures of \textit{nonparametric} estimations of dependence have been developed, such as distance correlation and kernel-based coefficients \citep{szekely2014partial,wang2015conditional,doran2014permutation}, but they tend to either only test for independence and not strength of functional relationship, or have complex instantations or asymptotic theories, making them difficult to deploy and rely upon in practice.

A recent coefficient, the Chatterjee coefficient, has been demonstrated to have a number of desirable properties, and has been further developed into an elegant method for testing conditional independence~\citep{chatterjee2020new,codec}. 
Particularly,
no conditional densities need to be estimated,
it can be computed in $O(n\log n)$ time where $n$ is the number of data samples,
it asymptotically to 0 for conditional independence, and 1 for measurable functions,
and it requires \textit{absolutely no assumptions} on the law over the random variables.
For arbitrary variables random $X,Y,Z$, where $Y$ is univariate and $X,Z$ can be multivariate of any size, the conditional coefficient is given  by
\begin{align}\label{eq:contcodec}
    T(Y,X|Z) = \frac{\int\EE\left[\VAR(\PP(Y\geq t)|X,Z)|Z)\right]d\mu(t)}{\int\EE\left[\VAR(\II\{Y\geq t\}|Z\right] d\mu(t)}
\end{align}
In Chapter~\ref{chap:lcodec} we will take advantage of these properties. Critically, without a specific procedure, identifying the conditioning set for a particular independence test is exponential. If we wish to find which variables $X_1,\ldots,X_n$ are sufficient for creating independence between some outcome $Y$ and the rest of the variables, na\"ively we would need to test all possible subsets.
An advantage of the coefficient of dependence in \cite{codec} is that it admits a linear time algorithm for iteratively building the sufficient set, naturally enabling an algorithm for constructing a Markov graph over the variables of interest.

\subsection{Hypothesis Testing}

Statistical hypothesis testing involves formally defining and testing a hypothesis about the world.
A prior ``null'' assumption is defined. The \textit{null} hypothesis, $H_0$
represents the default assumption or expectation
that there is no relationship or distinction among
the true population states,
whereas the \textit{alternative} hypothesis, $H_A$,
describes the world in which some hypothesized 
relationship or distinction does exist.
Testing proceeds by collecting observations
of the variables of interest,
computing a \textit{test statistic},
and comparing that statistic against
a prior \textit{null distribution}.
If the test statistic is larger
than a predefined critical value,
the null hypothesis is rejected:
there is reasonable evidence to suggest
the alternative may be true.
When we fail to reject the null,
there is insufficient evidence to support
the alternative claim.

The form of the test statistic and null distribution
are defined by the specific hypothesis being tested,
as well as the prior assumptions about the parameters
of interest and data collected.

\paragraph{Example: Testing a difference of means.} 
Say we have collected samples from two different groups,
representing the height of each person in the group.
Our task is to determine if the average height
of the two groups is significantly different from each other.
Na\"ively, we can compute the averages of the two groups and compare.
In practice these averages will never be equal, so how can we 
more rigorously determine if we should consider some measured difference
significant enough to say the groups are different?
We can set up the following hypothesis test.

Let $\mu_1$ be the true \textit{population mean} of group 1,
and $\mu_2$ the true population mean of group 2.
Our hypotheses are:
\begin{align}
H_0: \mu_1 = \mu_2 \qquad H_A: \mu_1 \neq \mu_2
\end{align}
Let us assume we have collected the same number of samples from each group,
$n$, and that the means of the collected samples are $\hat{\mu}_1$ and $\hat{\mu}_2$,
and the standard deviations are $s_1$ and $s_2$.
Then the test statistic
\begin{align}
t = \frac{\hat{\mu}_1 - \hat{\mu}_2}{\sqrt{\frac{s_1^2 + s_2^2}{n}}}
\end{align}
follows a $t$-distribution with $n-1$ degrees of freedom,
\textit{if} there is no difference between the means.
Comparing the value of the measured statistic over the observed data
to the corresponding $t$-distribution allows
us to determine how likely or unlikely it is that 
our observation follows the law that the means are equal.
If we want our test to accurately identify
a difference between means 95\% of the time,
we can set our threshold $t^*$ (critical value) for rejecting the null
to be the point where $\PP(t \geq t^*) \geq 0.95$.

Importantly, the value of the $t$-statistic and corresponding
distribution and critical value are extremely
dependent on the number of samples $n$ acquired for the test.
As we will see, in cases where the number of samples is 
very small and our hypothesis describes 
a subtle difference between groups,
novel tests and procedures are necessary
to effectively identify those differences.

\subsubsection{Likelihood ratios and permutation testing}
Two challenges often appear in practice:
\textbf{(1)} An obvious test statistic may not exist,
and \textbf{(2)} the null distribution of that statistic
or the one chosen may not have a clear form.

In the case of large parametric models,
distributions over hundreds of parameters or more
become infeasible to compute in practice. In these
cases, the \textit{likelihood-ratio test} can be used.
With our posterior notation above, we have the following
test statistic for the null hypothesis $\theta\in\Theta_0, \Theta_0\subseteq\Theta$ and the alternative $\theta\in\Theta^C:=\Theta\setminus\Theta_0$:
\begin{align}\label{eq:lrt}
\lambda_{LR} = -2\text{ln}\left[ \frac{\sup_{\theta\in\Theta_0}p(x|\theta)}{\sup_{\theta\in\Theta} p(x|\theta)} \right]
\end{align}
The true power of the LRT statistic comes from two fundamental results.
First, the distribution of $\lambda_{LR}$
asymptotically approaches a $\chi^2$ distribution
with a fixed number of degrees of freedom,
enabling easy testing (Wilk's Theorem, \citep{wilks}).
And second,
the Neyman-Pearson lemma states
that this likelihood-ratio test is the 
most powerful $\alpha$ level test for this case~\citep{neymanpearson}.

When even a likelihood ratio cannot be constructed in this form,
a nonparametric test exists that may be used
for any statistic defined by the problem of interest.
Consider again a test of the form in~\eqref{eq:hyptest},
but where instead of the mean we have some arbitrary
statistic over our two groups.
A \textit{permutation test} comprises
of generating a null distribution through
resampling. By shuffling the data
and recomputing the statistic,
we can estimate the distribution of the
statistic if there were no difference among the groups.
This distribution can then be used
as the null distribution for checking
if the true group allocations result in a 
test statistic significantly different
from the null distribution.

These ideas will be used to construct
and evaluate 
hypothesis tests in Chapter~\ref{chap:covtraj},
as well as in subsequent work building upon
those results.

\section{Differential Geometry}\label{sec:diffgeom}
Here we present a brief overview of differential geometry. For a more in-depth background, we refer interested readers to \cite{do1992riemannian}, \cite{lee2003smooth}, and \cite{spivak1981comprehensive}.


%Before discussion about general regressions models for manifold-valued response variables, let's revisit manifolds.
\textbf{Differentiable manifold.}
A \textit{differentiable (smooth) manifold} of dimension $n$ is a set $\Mc$ and a maximal family of \textit{injective} mappings $\varphi_{i}:U_{i}
\subset \textbf{R}^{n} \rightarrow \Mc$ of open sets $U_{i}$ of
$\RR^{n}$ into $\Mc$ such that:
\begin{enumerate}
\item $\cup_{i}\varphi_{i}(U_{i}) =\Mc$
\item for any pair $i,j$ with $\varphi_{i} (U_{i}) \cap
\varphi_{j} (U_{j}) = W \neq \phi$, the sets $\varphi_{i}^{-1}(W)$
and $\varphi_{j}^{-1}(W)$ are open sets in $\RR^{n}$ and the
mappings $\varphi_{j}^{-1} \circ \varphi_{i}$ are
differentiable, where $\circ$ denotes function composition. 
\item The family $\{(U_{i},\varphi_{i})\}$ is maximal relative to
the conditions (1) and (2). 
\end{enumerate}

Intuitively, a differentiable (smooth) manifold $\Mc$ is a topological
space that is locally similar to Euclidean space and has a globally
defined differential structure. 

\paragraph{Tangent space ($T_{p}\Mc$).} The \textit{tangent space} at $p \in \Mc$ is a vector space which consists of 
tangent vectors of \textit{all} possible curves passing through $p$. 

\paragraph{Tangent bundle ($T\Mc$).} The \textit{tangent bundle} of $\Mc$ is the disjoint union of tangent spaces at all points of $\Mc$, 
$T\Mc = \coprod_{p \in \Mc}T_{p}\Mc$. 
The tangent bundle is equipped with a natural \textit{projection map} $\pi: T\Mc \rightarrow \Mc$. 

\paragraph{Riemannian manifold.} A \textit{Riemannian manifold} is 
%a differentiable manifold which is 
equipped with a
smoothly varying metric (inner product), the \textit{Riemannian metric}. 

Various classical geometric notions, e.g., the angle between two curves 
or the length of a curve, can be extended to manifold spaces.

\paragraph{Geodesic curves.} A geodesic curve on a Riemannian manifold is the locally shortest (distance-minimizing) curve.
These are analogous to straight lines in Euclidean space 
and a main object to generalize linear models to Riemannian manifolds (as we will in Chapter~\ref{chap:covtraj}).

\paragraph{Geodesic distance.} The \textit{geodesic distance}
between two points on $\Mc$ is the length of the shortest {\em geodesic} curve connecting the two points. More generally, distance between two points on Riemannian manifolds is defined by the infimum of the length of all differentiable curves connecting the two points. Let $\gamma$ be a continuously differentiable curve $\gamma:[a,b] \rightarrow \Mc$ between $p$ and $q$ in $\Mc$ and $g$ be a metric tensor in $\Mc$.
Then, formally, the distance between $p$ and $q$ is defined as
\begin{equation}
\text{d}(p,q) := \inf_\gamma \int_a^b \sqrt{g_\gamma(t) (\dot{\gamma}(t), \dot{\gamma}(t))} dt
\end{equation}
where $\gamma(a)=p$ and $\gamma(b)=q$, and $\dot{\gamma}$ is the derivative or rate of change of $\gamma$.
%Such shortest curves are
%known as \textit{geodesic} and are analogous to straight lines in
%. 

\paragraph{Exponential map}. An exponential map is a map from a tangent space $T_p\Mc$  to $\Mc$, which is usually locally defined due to the existence and uniqueness of an ordinary differential equation for the map.
The geodesic curve from $y_i$ to $y_j$ can be parameterized by a tangent vector in the tangent space at $y_i$ with an exponential map $\EXP(y_i,\cdot ): T_{y_i}\Mc \rightarrow \Mc$. 

\paragraph{Logarithm map.}
The inverse of the exponential map is the \textit{logarithm map}, $\LOG(y_i,\cdot):\Mc \rightarrow T_{y_i}\Mc$. 
For completeness, Table \ref{tab:comp} shows corresponding operations in the Euclidean space and Riemannian manifolds.
In what follows, when operations are nested, the exponential map and its inverse logarithm map are denoted by $\EXP(p, x)$ and $\LOG(p, v)$ respectively, where $p, x \in \Mc$ and $v\in T_p\Mc$. They are usually denoted $\EXP_p(x)$ and $\LOG_p(v)$ in classical differential geometry literature. 
 
Separate from the above notations, the matrix exponential, i.e, $\exp(X):= \sum \frac{1}{k!} X^k$, where $0!=1$ and $X^0=I$  and the matrix logarithm are denoted fully lowercase by as $\exp(\cdot)$ and $\log(\cdot)$, similar to their univariate counterparts. These will be necessary in a specific instance in Chapter~\ref{chap:covtraj}, but otherwise will refer
to the common univariate functions in other contexts.
\begin{figure}
	\centering
	\includegraphics[width=\textwidth,trim={0 7cm 5cm 10cm},clip]{2_bknd/manifoldops.pdf}
	\caption[Operations on the Manifold]{From a basepoint $b$, we can identify the tangent space $T_b\cM$, use a direction $V\in T_b\cM$ to move using the tangent map, and use the exponential map to project back on to the manifold.}
	\label{fig:ot}
\end{figure}

% \renewcommand{\arraystretch}{1.5}
% \begin{table}
% {\footnotesize
% \begin{center}
%     \begin{tabular}{| l | l | l | }
%     \hline
%     Operation & Euclidean & Riemannian  \\  \hline 
%     \footnotesize Subtraction $\overrightarrow{x_i x_j}$& $x_j - x_i$ & $\LOG(x_i,x_j)$ \\ 
%     \footnotesize Addition $x_i+\overrightarrow{x_j x_k}$& $x_i + \overrightarrow{x_j x_k}$ & $\EXP(x_i,\overrightarrow{x_j x_k})$ \\     
%     \footnotesize Distance$(x_{i},x_{j})$  & $\| \overrightarrow{x_i x_j} \|$ & $\|\LOG(x_i,x_j) \|_{x_i}$ \\ 
%     Mean $\bar{x}$ & $\sum_{i=1}^{n} \overrightarrow{\bar{x}x_{i}}=0$ & \footnotesize $\sum_{i=1}^{n} \LOG(\bar{x}, x_i)=0$  \\ 
%     Covariance$(x)$ & \footnotesize$\E \left [ (x_i - \bar{x})(x_i - \bar{x})^{T} \right ]$&\footnotesize $\E \left [ \LOG(\bar{x}, x)\LOG(\bar{x}, x)^{T} \right ]$\\ [1ex] \hline 
%   \end{tabular}
% \end{center}
% }
% \caption{Basic operations in Euclidean space and Riemannian manifolds.}
% \label{tab:comp}
% \end{table}
\begin{table}
\begin{center}
    \begin{tabular}{| l | l | l | }
    \hline
    Operation & Euclidean & Riemannian  \\  \hline 
    \footnotesize Subtraction & $\overrightarrow{x_i x_j} = x_j - x_i$ & $\overrightarrow{x_i x_j} = \LOG(x_i,x_j)$ \\ 
    \footnotesize Addition & $x_i + \overrightarrow{x_j x_k}$ & $\EXP(x_i,\overrightarrow{x_j x_k})$ \\     
    \footnotesize Distance  & $\| \overrightarrow{x_i x_j} \|$ & $\|\LOG(x_i,x_j) \|_{x_i}$ \\ 
    Mean  & $\sum_{i=1}^{n} \overrightarrow{\bar{x}x_{i}}=0$ & \footnotesize $\sum_{i=1}^{n} \LOG(\bar{x}, x_i)=0$  \\ 
    Covariance & \footnotesize$\EE \left [ (x_i - \bar{x})(x_i - \bar{x})^{T} \right ]$&\footnotesize $\EE \left [ \LOG(\bar{x}, x)\LOG(\bar{x}, x)^{T} \right ]$\\ [1ex] \hline 
  \end{tabular}
\end{center}
\caption[Basic operations on Riemannian manifolds]{Basic operations in Euclidean space and Riemannian manifolds.}
\label{tab:comp}
\end{table}

\paragraph{Intrinsic mean.} 
Let $d(\cdot,\cdot)$ define the distance between two points. The intrinsic (or Karcher) mean is the minimizer to
\begin{equation}\label{eq:karchermean}
\bar{y} = \arg \min_{y \in \Mc} \sum_{i=1}^{N} d(y,y_{i})^{2}, 
\end{equation}
which may be an arithmetic, geometric or harmonic mean depending on $d(\cdot,\cdot)$.
A Karcher mean is a local minimum to \eqref{eq:karchermean},
and a global minimum is referred to as a Fr\'{e}chet mean.
On manifolds, the Karcher mean satisfies $\sum_{i=1}^{N} \LOG_{\bar{y}}y_i =0$.
\begin{algorithm}
	\caption{Karcher Mean on Manifolds}\label{alg:karcher}
	\SetAlgoLined
	\DontPrintSemicolon
	\KwIn{$y_{1}, \ldots, y_{N} \in \Mc$, $\alpha$}
	\KwOut{$\bar{y} \in \Mc$}
	\While{$ \| \sum_{i=1}^{N} \LOG(\bar{y}_{k},y_{i})\| > \epsilon$}{
		$\Delta\bar{y} = \frac{\alpha}{N} \sum_{i=1}^{N}\LOG (\bar{y}_k,y_i)$;\\
		$\bar{y}_{k+1} = \EXP(\bar{y}_k,\Delta \bar{y})$
	}
\end{algorithm}
This identity implies the first order necessary condition of \eqref{eq:karchermean},
i.e., $\bar{y}$ is a local minimum with a zero norm gradient \citep{karcher1977riemannian}.
In general, on manifolds, the existence and uniqueness of the Karcher mean is not guaranteed
unless we assume, for uniqueness, that the data is in a small neighborhood.

\paragraph{Parallel transport.} 
Let $\Mc$ be a differentiable manifold with an affine connection $\nabla$ and $I$ be an open interval.
Let $c:I \rightarrow \Mc$ be a differentiable curve in $\Mc$
and let $V_0$ be a tangent vector in $T_{c(t_0)}\Mc$, where $t_{0} \in I$. 
Then, there exists a unique parallel vector field $V$ along $c$, such that $V(t_0)=V_0$.
Here, $V(t)$ is called the \textit{parallel transport} of $V(t_0)$ along $c$. 

\subsection{Geometry of SPD manifolds}
As mentioned above, covariance matrices are symmetric positive definite matrices.
Here we focus our discussion to the above operations specific to SPD matrices.
 
Let SPD($n$) be a manifold for symmetric positive definite matrices of size $n\times n$.
This forms a quotient space $GL(n)/O(n)$, where
$GL(n)$ denotes the general linear group (the group of $(n \times n)$ nonsingular matrices)
and $O(n)$ is the orthogonal group 
(the group of $(n \times n)$ orthogonal matrices). 
Here, the tangent space $T_{p}\Mc$ is the space of symmetric matrices of dimension $(n+1)n/2$.

The inner product of two tangent vectors $u,v \in T_{p}\Mc$ is given by 
\begin{equation}
\begin{split}
  \langle u,v \rangle_{p} = \tr(p^{-1/2}up^{-1}vp^{-1/2})
\end{split}
\label{eq:metricSPD}
\end{equation}
This plays the role of the Fisher-Rao metric in the statistical model of multivariate distributions.
The geodesic distance is $d(p,q)^{2} = \tr( \log^{2}(p^{-1/2}qp^{-1/2}))$.

The exponential maps and logarithm maps are given as 
\begin{equation}
\begin{split}
  \EXP(p,v) = p^{1/2} \exp(p^{-1/2}vp^{-1/2})p^{1/2}, \;\;
  \LOG(p,q) = p^{1/2} \log(p^{-1/2}qp^{-1/2})p^{1/2}.
\end{split}
\end{equation}
Let $p, q$ be in SPD($n$) and a tangent vector $w \in T_{p}\Mc$, the
tangent vector in $T_{q}\Mc$ which is the parallel transport of $w$ along
the shortest geodesic from $p$ to $q$ is given by 
\begin{equation}
\begin{split}
\Gamma_{p \rightarrow q}(w) &= p^{1/2}rp^{-1/2}wp^{-1/2}rp^{1/2} \\
\text{where } r &= \exp \left (p^{-1/2}\frac{v}{2}p^{-1/2} \right ) \text{ and
}v = \LOG(p,q)
\end{split}
\end{equation}

\paragraph{Other spaces of interest.}
Orthogonal matrices of fixed size and rank also form a manifold, the (compact) \textbf{Stiefel Manifold}: $ \ST(p,n)=\left\{Y \in \RR^{n\times p} | Y^TY=I_p,\: p \leq n\right\}$.
An arbitrary $X \in \RR^{n \times p}$ matrix can be projected onto the Stiefel manifold $\ST(p,n)$ using $X \mapsto UV^T$ where $X=U\Sigma V^T$ is the (thin) singular value decomposition of $X$. We will use this fact in Chapter~\ref{chap:ott}.

\subsection{Tensors and their Geometry}
Let $\cX \in \RR^{n_1 \times \cdots \times n_d}$ be a $d$-dimensional array, or tensor, with each mode having length $n_i$. To store a full rank tensor, $n^d$ storage would be required. A number of tensor factorizations have been developed to reduce this storage cost. The CANDECOMP/PARAFAC (CP) \cite{harshman1970foundations,carroll1970analysis} decomposition reduces the storage to $O(dnr)$, but finding the exact CP-rank $r$ is NP-hard. Hierarchical tensor methods have also proven to be effective in tensor compression \cite{cohen2016expressive, cohen2016convolutional}. {\color{red} maybe slow down here, expand a bit}

A more recent decomposition, the \textit{Tensor Train} decomposition (TT) \cite{oseledets2011tensor}, defines an element of the tensor as
\begin{align}
\cX(x_1,\ldots,x_d) = A_1(x_1)\cdots A_d(x_d)
\end{align}
where $x_i \in \left\{1,\ldots, n_i\right\}$, and $A_i(x_i) \in \RR^{r_{i-1} \times r_{i}}$ for each $i \in \{1,\ldots,d\}$ are called the \textit{cores} of the tensor train, with $r_0 = r_d = 1$. Equivalently, the full tensor is written as:
\begin{align}\label{eq:fullTT}
	\cX &= \sum_{k_0 = 1}^{r_0} \cdots \sum_{k_d = 1}^{r_d} A_1(k_0,:,k_1) \otimes \cdots \otimes A_d(k_{d-1},:,k_d) 
\end{align}
where $A_i \in \RR^{r_{i-1}\times n_i \times r_i} $. This format requires $O(dnr^2)$ storage, but has two major advantages over the CP format. First, finding the TT-rank (the smallest set of $r_i$'s that satisfy the decomposition with equality) of any arbitrary tensor is tractable, and as such all tensors can be efficiently rewritten in the TT format. Second, projecting arbitrary tensors onto the TT format of a fixed rank requires only a set of QR and singular value decompositions \cite{oseledets2011tensor}. This projection, \textit{TT-rounding}, additionally allows for a given TT tensor of some rank to be projected onto the space of TTs with lower rank, and requires $O(dr^3)$ computational complexity. Separately, specific tensor train constructions have recently been identified as forms of general recurrent networks \cite{khrulkov2018generalized}.

We denote a tensor operator $\tG$ as a grouping of tensor modes into an ``input" and ``output" list, such that $\tG \in \RR^{(n_1^{in},\times\ldots\times,n_d^{in}) \times (n_1^{out},\times\ldots\times,n_d^{out})}$. This operator $\tG$ can be seen as the TT representation of a matrix $W \in \RR^{(n_1^i\cdots n_d^i) \times (n_1^o\cdots n_d^o)}$. In \cite{novikov2015tensorizing}, authors use this formulation to directly compress the weight layers in neural networks. Cores in the operator are indexed by both an input and output index, i.e., $A_i(x_i,y_i) \in \RR^{r_{i-1}\times r_i}$, where $x_i \in [1,\ldots,n_i^{in}], y_i \in [1,\ldots,n_i^{out}]$.

Common operations upon tensor trains require \textit{matricizing} the cores of the TT format. Here, we define the left matricization of core $A_i(x_i)$ as $A_i^L \in \RR^{r_{i-1}n_i \times r_i} $ and the right matricization similarly.

{\color{red} example operations, add/mult, QR decomp}

\subsection{Differential Geometry of Tensor Trains}
Tensor trains with fixed TT-ranks form a Riemannian submanifold of $\RR^{n_1 \times \cdots \times n_d}$ \cite{lubich2015time, holtz2012manifolds}:
\begin{align}\label{eq:riem}
	\cM_r := \{ \cX \in \RR^{n_1 \times \cdots \times n_d} \text{ with TT-ranks } r_0,\ldots,r_d\} 
\end{align}
Optimizing a function with respect to a Riemannian manifold-valued variable amounts to computing a free derivative in the ambient space, projecting the gradient to the tangent space of the current iterate, and using the (retraction) exponential map to compute the next iterate.
The authors in \cite{novikov2016exponential} use this procedure to more effectively learn a model of all exponentially many interactions in a linear model.


\section{Deep Networks, Optimization, and Objectives}
The form of the function $f_\theta$ in~\eqref{eq:learning}
is critical in determining both
the types of optimization methods
that may identify a solution,
and the particular minimizer identified.
In the learning methods that follow,
$f$ will typically take the form of 
a deep neural network.
The advantages
and successes of deep neural networks
rely heavily on their ease of optimization:
the \textit{computation graph} that 
underlies the neural network
allows for gradients
to be computed by parts
and accumulated via the chain rule.

Consider a simple function $f_\theta(x)$
that is defined as a linear combination of 
some parameters $\theta:=w, w\in \RR^d$ with $x\in \RR^d$ followed by 
a differentiable, nonlinear scalar \textit{activation} function $a(\cdot)$:
\begin{align}\label{eq:wx}
	f_\theta(x) := a(w\cdot x)
\end{align}
If we have some estimate of the parameters $\theta:=w$,
then the gradient of the full network with respect to those parameters is
\begin{align}
	\frac{df}{d\theta} = \frac{df}{da}\frac{da}{dw}
\end{align}
where $df/da$ is the (known) derivative of the \textit{activation} function,
and $da/dw$ is exactly $x$, the derivative of a linear function.
With a direction of descent,
we can update the parameters via some update to minimize the functional $f$ of interest:
\begin{align}\label{eq:fullgd}
	\theta_{t+1} = \theta_t + g(\theta_t,x)
\end{align}
where $g(\cdot)$ is some function of the full derivative $g(\cdot) := g(\nabla f_\theta(x))$,
and $\theta_t$ are the current parameter estimates.
Optimization proceeds and terminates when a certain amount
of iterations $t$ have completed,
or some stopping criterion has been reached,
typically that the gradient is small,
indicating that a minima has been identified.

In data science and machine learning applications,
we typically do not have a fixed $x$, but 
rather a dataset $X:=\{x_i\}_{i=1}^n$.
If the dataset is small,
we may be able to still compute an update as in~\eqref{eq:fullgd}.
But this is infeasible when we have thousands,
hundreds of thousands, or millions of samples.
In this case,
stochastic gradient descent (SGD) is used.
Gradient updates in SGD proceed
by taking a single sample and computing~\eqref{eq:fullgd}.
\textit{Training} of $\theta$
follows by iteratively updating the parameters
over full passes of the dataset.
When feasible, mini-batches of samples can
be used instead of a single sample,
and in both cases convergence and convergence
rates have been shown to be reasonable~\citep{hardt2016train}.

\paragraph{Hessians.}
Uninformed gradient updates
are generally preferred
for their speed and ease of computation.
However, additional information in the form of the \textit{Hessian}
can lead to faster convergence
as well as a number of theoretical properties and guarantees.
Consider the function at a critical point $\theta^*$.
The Taylor expansion of the function at that point is
\begin{align}
f(\theta) = f(\theta^*) + \nabla f(\theta^*)^\top f(\theta - \theta^*) + \frac{1}{2}(\theta - \theta^*)^\top H(\theta^*)(\theta - \theta^*) + \ldots
\end{align}
Where $H(\theta^*)$ is the Hessian matrix at the point $\theta^*$. 
With no additional terms, this second-order approximation provides information
about the local curvature of the function near the critical point,
allowing a scaling of the gradient that can use this local 
curvature to inform optimization:
\begin{align}\label{eq:newtonstep}
	\theta_{t+1} = \theta_t + H(\theta_t)^{-1} g(\theta_t)
\end{align}
These Newton updates are typically infeasible in most
machine learning applications with high-dimensional
parameter spaces, where the complexity of
the actual function or the maximal moment is unknown.
In some cases, Hessian approximations,
or its eigenspectrum can be efficiently computed,
and as we will see this can lead to 
practical measures that lead to new
algorithms and guarantees.

For more on these ideas, 
and a formal treatment with respect to 
general optimization, see~\cite{wright1999numerical}.

These formulations and SGD updates generalize
to extremely large and complex stacks
of operations, and are what have enabled
the enormous success and ubiquity of learning
methods.
While ``fully-connected" layers, such as
in~\eqref{eq:wx} are the simple in their form,
they have been proven to be sufficient
in large capacity to serve as
\textit{universal function approximators}~\citep{abc}.
In this form however, capacity guarantees
require the number of parameters (dimension of $w$)
to grow exponentially: infeasible in practice.
If the size is misspecified, training
can lead to parameter settings that are provably
optima at that level, but fail to sufficiently
capture the true problem complexity.
These problems extend to stacks of ``fully-connected"
layers as well: nonlinearity through multiplication
and activations is effective, but leads
to poor training time results due to 
the speed at which a local minima can be found~\citep{abc}.

The final minima identified 
can vary significantly based on 
the particular form of the function $f_\theta$, 
and, in the case where the function $f$ is not \textit{convex}, 
it can also depend on the estimate $\theta_0$ at initialization.
This detail has been used to suggest
that large, complex neural networks 
may contain sub-networks that can be trained in isolation
to perform a task with high accuracy,
even when randomly initialized.
The ``lottery ticket hypothesis" states 
a small, sparse sub-network can be trained to perform just as well as the larger network, 
but with fewer parameters and less computation.
\begin{figure}
	\includegraphics[]{example-image-a}
	\caption[Initialization importance in nonconvex problems]{Nonconvex functions optimized using variations of gradient descent can lead to different (and potentially non-global) optima.}
\end{figure}

A wide variety of neural network \textit{architectures}
have been proposed depending on the type of data and learning task.
Some of the most commonly used architectures include
Convolutional Neural Networks (CNNs), Recurrent Neural Networks (RNNs),
Transformers, Autoencoders, and Generative Adversarial Networks (GANs).
CNNs, mainly used in image recognition and processing tasks,
use convolutions designed to process grid-like imaging data ~\citep{abc}. 
RNNs are designed to handle sequential data such as time series or language,
and have a memory-like mechanism that allows information to persist in hidden state vectors~\citep{abc}. 
Transformers, now the de-facto method in Natural Language Processing (NLP),
make use of a self-attention mechanism,
weighing the importance of different input tokens for eventual prediction~\citep{abc}. 
Autoencoders and Generative Adversarial Networks (GANs) fall under the category of generative models,
trained to generate outputs that resembles real data~\citep{abc}. 
Newer generative models built on diffusion methods have 
been able to produce photorealistic images based
only on captions input by the user~\citep{abc}.
All of these methods come with their own
idiosyncracies in model capacity and practical
methods for efficient training.

With these varying architectures
has also come the field of \textit{architecture search},
identifying the hyperparameters and layer types
that would lead to the best downstream performance
with the most efficient full architecture.
While computationally expensive, techniques
such as reinforcement learning, evolutionary algorithms,
gradient-based methods, and even greedy approaches
have been used to identify state-of-the-art networks~\citep{abc}.

\subsection{Losses and Probability Measures}
\todo{losses/prob. measures}
The methods for optimization above
lend themselves to typically arbitrary functions.
As described in the Introduction,
typically a \textit{loss function}
is defined to measure the disparity
between the prediction or output of a model
and the target of interest.
Optimizer flexibility has led
to the development and design of loss functions
that suit particular tasks,
or those that correspond more directly
to practitioners' high-level goals.
Building on the classical mean-squared error:
\begin{align}
l(f_\theta(x), y) := (y - f_\theta(x))^2
\end{align}
methods have extended to information-based schemes
such as cross-entropy, as well as 
incorporating classical \textit{regularization} schemes
to push solutions towards desirable minima.
These typically take the form of additive terms 
penalizing large norms over weights, where
the norm chosen corresponds to the choice of 
prior assumed by the user.

Following the information-based schema,
significant work has been done
on probabilistic forms of losses,
treating the input and output spaces
of models as distributions.
Particularly useful for generative models,
$f$-divergences have been studied
as a general form of distances
between probability distributions
that can be effectively minimized to
train in such a way that model outputs
come from a maximum-entropy distribution
with respect to the original training data~\citep{fgan}.
Measures such as KL-divergence, mutual information,
maximum-mean discrepency, and others all fall within
this framing.

\subsection{Optimal Transport}	
Of particular interest in this thesis is the \textit{Wasserstein} distance,
or traditionally known as the Earth Mover's distance.
Recent developments in GANs \citep{wgan} have demonstrated
the Wasserstein metric is typically more stable
compared to other measures and avoids
\textit{mode collapse}, sharp local minima 
with low variation. Interestingly
this formulation can be viewed as an approximation
of the \textit{optimal transport} problem.

Optimal transport refers to the problem of finding
a transportation plan that minimizes some cost of transforming
one probability distribution to another.
Developments in Riemannaian geometry and measure theory 
have led to a general formulation.
\begin{definition}[Monge-Kantorovich Problem \citep{mongekant}]\label{def:mongekant}
	Let $\mu,\nu$ be probability measures over separable metric spaces $X$ and $Y$.
	The optimal transportation problem seeks to find a joint measure $\gamma$ on $X\times Y$
	that satsifies
	\begin{align}
	\gamma^* = \text{inf} \left\{\left. \int_{X\times Y} c(x,y) d\gamma(x,y) \right| \gamma \in \Gamma(\mu,\nu) \right\}
	\end{align}
\end{definition}
Where $\Gamma$ is the space of all probability measures with marginals equal to $\mu$ on $X$ and $\nu$ on $Y$.
More common in recent machine learning is the \textit{Wasserstein} distance,
defined as the $p$-th distance over the Monge-Kantorovich problem~\citep{rubenstein}.
\begin{definition}[Wasserstein metric]\label{def:wassmetric}
	The Wasserstein $p$-distance is given by:
	\begin{align}
	W_p(\mu, \nu) = \left( \inf_{\gamma \in \Gamma(\mu, \nu)} \EE_{(x, y) \sim \gamma} d(x, y)^p \right)^{1/p}
	\end{align}
\end{definition}
And the discrete analog:
\begin{definition}[Earth Mover's Distance]
	Let $p_1$ and $p_2$ be distributions with discrete support of size $n$, and $x(i,j)$, where  $x\in\RR^{n\times n}$, denotes the movement of ``mass" from $p_1(i)$ to $p_2(j)$.
	Denote by $c(i,j)$ the cost of moving one unit of mass from  $p_1(i)$ to $p_2(j)$.
	The Earth Mover's Distance (EMD) between $p_1$ and $p_2$ is the minimal cost to transform $p_1$ into $p_2$.
	%given by the sum of the costs associated with shifting mass, according to $x(i,j)$ such that $p_1$ is transformed into $p_2$.
	% Unless otherwise specified, we assume $c(i,j)=|i-j|$, which corresponds to ground distance. 
	%This can be 
	Written as a linear program (LP):
	\begin{align}\label{eq:2d-emd}
	\begin{aligned}
	\underset{{x\in \RR^{n\times n}_+}}{\textrm{min}} \sum_{i,j} c(i,j) x(i,j) \quad  \textrm{s.t.}\quad \sum_j x(i,j) &= p_1(i); \ 
	\sum_i x(i,j) = p_2(j),\ (\forall i,j\in[n]).
	\end{aligned}
	\end{align}
\end{definition}
Computation of the continuous measures can be straightforward with some assumptions,
leading to natural linear programming formulations akin to the EMD.
However in cases where one wishes to \textit{minimize} this distance,
practical complexity explodes, even in the discrete
formulations commonly found in application.
A new entropic regularization method introduced in \cite{lightspeed}
has led to newfound interest,
use, and analysis of optimal transport for deep learning applications.
In Chapter~\ref{chap:demd} we will expand upon these
ideas to the case where we have multiple distributions,
and wish to minimize the distance among all of them concurrently.
\todo{OTfig}    


Each chapter in the sequel
is defined by a unique intersection
of the above ideas, leading to new insights
with respect to the subset selection problem of interest.
The relevant background will be referenced and refreshed 
as needed to ease narrative continuity.