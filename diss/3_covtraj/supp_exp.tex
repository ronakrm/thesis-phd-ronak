\subsection*{Detailed Imaging with Cognitive Tests Results}
In the following tables we provide additional details of the statistical test we performed on the preclinical AD cohort. Each set contains a set of features found to display significant group difference (at the $p \leq 0.05$ level) along the covariance trajectory divided by the group variable indicated.

While some of these associations are well-known, few have been indicated as novel by AD researchers and clinicians, and to be of interesting value for further analysis. 
%Even though it is possible that these subsets are supersets of the true feature set, traditional modeling methods may provide additional insight now that the number of features to be considered has been reduced significantly.

\begin{table}[!]
	\centering
	\begin{tabular}{p{0.8cm}p{5.5cm}p{6cm}}
		\toprule
		\multicolumn{3}{c}{\textbf{Amyloid Load (PiB Positivity)}}\\ \midrule \midrule
		%		    \multicolumn{3}{c}{Gender}}                   \\\
		Set 1 & PiB Angular L/R & PiB Cingulum Ant L/R \\
		& PiB Cingulum Post L/R & PiB Frontal Med Orb L/R \\
		& PiB Precuneus L/R & PiB Temporal Sup L/R \\
		& PiB Temporal Mid L/R & \textbf{PiB SupraMarginal L} \\
		\midrule
		Set 2     & FA Cerebral peduncle R   & FA Cerebral peduncle L	\\
		& MD Corticospinal tract R	& MD Corticospinal tract L		\\
			     & Trail-Making Test Part A Score  & MD Cerebral peduncle R \\ 
			    &PET Cingulum Post R  &  \\ \midrule\bottomrule
	\end{tabular}
	\caption{Group difference across Amyloid Load (PiB Positivity)}
\end{table}

\begin{table}[!]
	\centering
	\begin{tabular}{p{0.8cm}p{5.5cm}p{6cm}}
		\toprule
		\multicolumn{3}{c}{\textbf{Gender}}\\ \midrule \midrule
        
		Set 1     & Rey Audio and Verbal Learning Test   & FA Cingulum  L	\\
		\midrule
		   & FA Medial lemniscus L	& FA Cingulum (hippocampus) L		\\
		Set 2    & FA Posterior thalamic radiation \newline (include optic radiation) L& \\ \midrule
    	Set 3    & FA Corticospinal tract R& FA Superior fronto-occipital fasciculus  R \\	\midrule\bottomrule
	\end{tabular}
    \caption{Group difference in gender}
\end{table}

\begin{table}[!]
	\centering
	\begin{tabular}{p{0.8cm}p{5.5cm}p{6cm}}
		\toprule
		\multicolumn{3}{c}{\textbf{Genotype: APOE4}}\\ \midrule \midrule
		Set 1    &Digit Span Backward Raw Score & Stroop Color-word \\ 
		 &PiB Cingulum Post L &    PiB Cingulum Post R         \\
		&PiB Frontal Med Orb L & 		PiB Frontal Med Orb R \\ 
		&PiB Precuneus L & PiB Precuneus R \\ 
		& PiB SupraMarginal &  PiB Temporal Mid R \\ \midrule		 \bottomrule
	\end{tabular}
	\caption{Group difference across Genotype APOE4 expression}
\end{table}

\begin{table}[!]
	\centering
	\begin{tabular}{p{0.8cm}p{5.5cm}p{6cm}}
		\toprule
		\multicolumn{3}{c}{\textbf{Consensus Conference}}\\ \midrule \midrule
		Set 2    &Digit Span Backward Raw Score & Stroop Color-word \\ 
		 &PiB Cingulum Post L &    PiB Cingulum Post R         \\
		&PiB Frontal Med Orb L & 		PiB Frontal Med Orb R \\ 
		&PiB Precuneus L & PiB Precuneus R \\ 
		& PiB SupraMarginal &  PiB Temporal Mid R \\ \midrule
		\bottomrule
	\end{tabular}
	\caption{Group difference across Expert MCI Diagnosis}
\end{table}

%\subsubsection*{Motivating examples to seek group differences in temporal trends}
%In Figure \ref{fig:imagehists} we present histograms detailing the distribution of two critical cognitive tests, stratified across various groups of scientific interest. These distributions were the key motivation for our exploration into the methods described in the paper, as is demonstrated by the small differences in mean across groups regardless of grouping selection, and the saturation that occurs at the ceiling of the cognitive tests because of the high cognition overall across the cohort.
%
%\begin{figure}[bh!]
%\centering
%\includegraphics*[width=0.8\textwidth]{BNTHists.pdf}
%\includegraphics*[width=0.8\textwidth]{TTOTALHists.pdf}
%\caption{Histograms of the Boston Naming Test Scores and RAVLT Total Scores for all time points for the $114$ individual measurements across different group separations.}
%\label{fig:imagehists}
%\end{figure}
