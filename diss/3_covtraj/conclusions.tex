The analysis of datasets to identify where clinically disparate groups differ is pervasive in biology, neuroscience, genomics and epidemiological studies. 
We find that graphical models are an ideal tool to analyze high-dimensional data in these areas but have been sparingly used for the analysis of 
group-wise differences, especially in a longitudinal setting. 
Motivated by an application related to longitudinal analysis of imaging and clinical/cognitive data from otherwise healthy individuals 
who are at risk for Alzheimer's disease (AD), we show how a combination of manifold regression with a generalization of scan statistics to the graph setting yields 
tools that can be directly deployed. 
We present an efficient algorithm and develop the theoretical results showing the regimes where its application is appropriate. 
In various experiments, while the standard schemes are not sufficiently powered to detect the signal, our proposed formulation is able to 
detect meaningful group difference patterns, many of which have a clear scientific interpretation. 
We believe that these results are promising for the neuroimaging application 
described and other regimes where group-wise analysis is desired but the number of features is large. 