\begin{acks}
% meal replacements, madison metro
% drugs/therapy
% family
% vikas
% beth, angela, desmond, pradeep
% michael newton
% sterling
% commitee fred yongjae 


I was lucky enough to work with a number of amazing labmates, collaborators, and friends. In no particular order, the following people have contributed immensely via collaborations, discussions, and support in countless ways.
Hyunwoo Kim, Seong Jae Hwang, Zirui Tao, Tianyi Shan, Haoliang Sun, Jurijs Nazarovs, Anita Sinha, Jeffery Kline, Sourav Pal, Vishnu Lokhande, Zhichun (Eric) Huang, Jeffery Kline, Sathya Ravi, Yunyang Xiong, Rudrasis Chakraborty,

Won Hwa Kim, Vamsi Ithapu, Takis Chytas, Xingjian Chen, Zihang Meng were great labmates, and though we did not collaborate directly contributed significantly to making the lab a good place to work.

The following faculty were immensely helpful in providing both guidance and direct collaboration, and were it not for them this process may have been significantly more painful.
Sterling Johnson, Michael Newton, Glenn Fung. 
I'm also thankful for the advice on specific projects given by Vivek Prabhakaran, Ming Yuan, and Sterling Johnson, as well as my committee for providing helpful feedback as this thesis developed.

The following friends, peers, and roommates contributed significantly to the atmosphere, and experience,
and going home to and out with them at the end of a week or after a deadline was crucial to being able to do it again for the next. Michael O'Neill, Christopher Magnano, Ross Kleiman, Bryce Sandlund, Hollis Howe.

Therapy and medication aside, this would not have been possible without the continuous and unwavering support of my advisor Vikas. There were many times where I was uncertain, struggling, or concerned,
and advice and discussions with him helped alleviate those feelings. Aside from the explicit and direct collaborations, either through whiteboard discussions or publication writing, those higher-level, personal conversations were crucial in carrying me over the finish line.

And lastly this work could not have been completed
without a number of grants supporting
myself and my collaborators.
Work in Chapter 3 was supported in part by NIH grants R01 AG040396, AG021155, EB022883 
and NSF grants DMS 1265202 and CAREER award 1252725.
The authors were also supported by the \href{http://cpcp.wisc.edu/}{UW Center for Predictive Computational Phenotyping} (via BD2K award AI117924) and the \href{http://www.adrc.wisc.edu/}{Wisconsin Alzheimer's Disease Research Center} (AG033514). 
I was supported by a fellowship via training grant award T32LM012413.
Work in Chapter 4 was supported by grants NSF CAREER award RI 1252725, UW CPCP (U54AI117924), R01AG059312, R01EB022883, RF1AG062336, 
and a NIH predoctoral fellowship to RM via T32 LM012413.
Work in Chapter 5 was supported by grants from the National Institutes of Health numbered RF1AG059312, RF1AG062336 and RF1AG059869, NSF award CCF 1918211, funds from the American Family Insurance Data Science Institute at UW-Madison, and UIC-ICR start-up funds.
Work in Chapter 6 was supported in part by NIH grants RF1AG059312, RF1AG062336, RF1AG059869, and NSF award CCF 1918211.
The work presented here was partially developed while myself and collaborators Jeffery Kline and Glenn Fung were at the Machine Learning Group at American Family Insurance.

\end{acks}
