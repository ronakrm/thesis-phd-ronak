\chapter{Follow-up and Future work}\label{chap:discuss}

The work presented here has been and continues 
to be extended in a number of different theoretical 
and empirical directions.

Work in Chapters~\ref{chap:covtraj} and \ref{chap:ott} has been further developed for a number of other applications.
Particularly, the covariance trajectory analysis
methods have been extended 
to the analysis of other Alzheimer's Disease populations~\citep{isbi},
and work on understanding and localizing temporal lobe epilepsy
measured over resting-state functional MRI acquisitions is upcoming.
Ideas from both chapters have contributed
to the understanding and modeling of uncertainty
in recurrent models~\citep{spgru},
and more generally towards efficient
Bayesian methods for deep Monte Carlo methods~\citep{mcreparam}.
Applications based on modeling disease progression
as a function of both static and time-varying 
variables was extended to ``deepify-ing" mixed effects models~\citep{deepmem}.

Work in unlearning in Chapter~\ref{chap:lcodec} is relatively nascent,
but excitement in the field is growing
and a number of works have been motivated
by recent successes.
Methods for directly measuring, and perhaps
even promoting or reducing conditional
independence are beginning to be explored.
Building upon these ideas,
we briefly describe a particular 
direction of interest, motivated
by the recent large successes of large language models.

\section{Interpretability and Conditional Independence In Deep Latent Spaces}\label{sec:latents}
\todo{ongoing/future work on latent spaces and condindep}

