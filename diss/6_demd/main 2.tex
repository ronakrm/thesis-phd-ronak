
\documentclass{article} % For LaTeX2e
\usepackage{iclr2023_conference,times}

% Optional math commands from https://github.com/goodfeli/dlbook_notation.
\input{math_commands.tex}

\usepackage{hyperref}
\usepackage{url}

%%% Personal Packages

\usepackage{booktabs}       % professional-quality tables
\usepackage{amsfonts}       % blackboard math symbols
\usepackage{nicefrac}       % compact symbols for 1/2, etc.
\usepackage{microtype}      % microtypography
\usepackage{xcolor}         % colors

\usepackage{graphicx}
\usepackage{caption}
\usepackage{wrapfig}

% For theorems and such
\usepackage{amsmath}
\usepackage{amssymb}
\usepackage{mathtools}
\usepackage{amsthm}

% if you use cleveref..
\usepackage[capitalize,noabbrev]{cleveref}

%%%%%%%%%%%%%%%%%%%%%%%%%%%%%%%%
% THEOREMS
%%%%%%%%%%%%%%%%%%%%%%%%%%%%%%%%
\theoremstyle{plain}
\newtheorem{theorem}{Theorem}[section]
\newtheorem{proposition}[theorem]{Proposition}
\newtheorem{lemma}[theorem]{Lemma}
\newtheorem{corollary}[theorem]{Corollary}
\theoremstyle{definition}
\newtheorem{definition}[theorem]{Definition}
\newtheorem{assumption}[theorem]{Assumption}
\theoremstyle{remark}
\newtheorem{remark}[theorem]{Remark}

% Todonotes is useful during development; simply uncomment the next line
%    and comment out the line below the next line to turn off comments
% \usepackage[disable,textsize=tiny]{todonotes}
% \usepackage[textsize=tiny]{todonotes}

\usepackage{algorithm}
\usepackage{algpseudocode}
\algtext*{EndWhile}% Remove "end while" text
\algtext*{EndFunction}% Remove "end while" text
\algtext*{EndFor}% Remove "end while" text
\algtext*{EndIf}% Remove "end if" text


\newtheorem{theorem}{Theorem}
\newtheorem{lemma}{Lemma}
\newtheorem{proposition}{Proposition}
\newtheorem{definition}{Definition}
\newtheorem{corollary}{Corollary}
\newtheorem{remark}{Remark}
\newtheorem{assumption}{Assumption}

\def\EE{\mathbb{E}}
\def\II{\mathbb{I}}
\def\PP{\mathbb{P}}
\def\RR{\mathbb{R}}


\def\cA{\mathcal{A}}
\def\cD{\mathcal{D}}
\def\cH{\mathcal{H}}
\def\cI{\mathcal{I}}
\def\cL{\mathcal{L}}
\def\cM{\mathcal{M}}
\def\cP{\mathcal{P}}
\def\cS{\mathcal{S}}
\def\cT{\mathcal{T}}
\def\cU{\mathcal{U}}
\def\cW{\mathcal{W}}
\def\cX{\mathcal{X}}
\def\cY{\mathcal{Y}}
\def\cZ{\mathcal{Z}}

\newcommand\myas{\stackrel{\mathclap{\normalfont\mbox{a.s.}}}{=}}

\DeclareCaptionLabelFormat{andtable}{#1\ #2}

% for comments during writing, custom
\newcommand{\authnote}[3]{\textcolor{#2}{\textbf{#1:} #3}}
\newcommand{\vikas}[1]{{\authnote{Vikas}{blue}{#1}}}
\newcommand{\ronak}[1]{{\authnote{Ronak}{blue}{#1}}}
\newcommand{\vishnu}[1]{{\authnote{Vishnu}{blue}{#1}}}
\newcommand{\jeff}[1]{{\authnote{Jeff}{blue}{#1}}}
\newcommand{\glenn}[1]{{\authnote{Glenn}{blue}{#1}}}

\title{Efficient Discrete Multi-Marginal \\ Optimal Transport Regularization}

\author{%
  David S.~Hippocampus\thanks{Use footnote for providing further information
    about author (webpage, alternative address)---\emph{not} for acknowledging
    funding agencies.} \\
  Department of Computer Science\\
  Cranberry-Lemon University\\
  Pittsburgh, PA 15213 \\
  \texttt{hippo@cs.cranberry-lemon.edu} \\
  % examples of more authors
  % \And
  % Coauthor \\
  % Affiliation \\
  % Address \\
  % \texttt{email} \\
  % \AND
  % Coauthor \\
  % Affiliation \\
  % Address \\
  % \texttt{email} \\
  % \And
  % Coauthor \\
  % Affiliation \\
  % Address \\
  % \texttt{email} \\
  % \And
  % Coauthor \\
  % Affiliation \\
  % Address \\
  % \texttt{email} \\
}

\usepackage{lipsum}  

\begin{document}

\maketitle
\begin{abstract}
Optimal transport has emerged as a powerful tool for a variety of problems in machine learning, and it is frequently used to enforce distributional constraints.
In this context, existing methods often use either a Wasserstein metric, or else they apply concurrent barycenter approaches when more than two distributions are considered.
In this paper, we  leverage multi-marginal optimal transport (MMOT), where we take advantage of a procedure that computes a generalized earth mover's distance as a sub-routine. 
We show that not only is our algorithm computationally more efficient compared to other barycentric-based distance methods, but it has the additional advantage that gradients used for backpropagation can be efficiently computed during the forward pass computation itself, which leads to substantially faster model training. We provide technical details about this new regularization term and its properties, and we present experimental demonstrations of faster runtimes when compared to standard Wasserstein-style methods. Finally, on a range of experiments designed to assess effectiveness at enforcing fairness, we demonstrate our method  compares well with alternatives.

%\glenn{I massaged the abstract a little}
%which are effective when distributions are continuous and known, or when measures of interest are discrete.
% Our formulation allows for a discretization of continuous measures that drop in directly to classical  formulations of the Earth Mover's Distance. 
% We present an alternative distance measure based on Earth Mover's distance to practically and efficiently optimize with distributional constraints within typical machine learning models. 
%  {\color{blue}JK: i don't understand this sentence - modify or remove.  We measure these constraints over outputs in neural network pipelines prior to thresholding, allowing for optimization over multiple thresholds, or operating points, chosen a posteriori. }
% Fairness of predictive models is perhaps the most important problem preventing their widespread adoption in social settings. New ways of addressing fairness have been developed, by identifying and minimizing various metrics between distributions over groups. While these are typically easy to compute for 2 groups, existing methods for computation via Wasserstein barycenters and Sinkhorn regularization scale poorly with many groups. Furthermore, they are limited in working only on binarized outcomes. Using a new algorithm for the computation of Earth Mover's Distance, we instantiate fairness over a discretization of a continuous measure \textit{prior to thresholding}, and compute a fast, differentiable measure of fairness over it. This method allows for fast in the loop regularization, and provides a final model that is fair across all discretizations. We provide technical results defining the measure and its properties, and follow with experimental demonstrations of speedup over existing methods and ability to construct fair models.
\end{abstract}

% \listoftodos

\section{Introduction}
The use of Optimal transport (OT) is now prevalent in many problem settings including information retrieval \citep{balikas2018cross,yurochkin2019hierarchical}, image processing \citep{otip}, statistical machine learning, and more recently, for ethics and fairness research \citep{kwegyiraggrey2021relative}. 
OT is well-suited for tasks where dissimilarity between two or more probability distributions must be quantified; 
its success was made possible through dramatic improvements in algorithms \citep{cuturi2013sinkhorn,solomon2015convolutional} that allow one to efficiently optimize commonly used functionals.
In practice, OT is often used to estimate and minimize the 
distance between certain (data-derived) distributions, 
using an appropriately defined loss functional. %Recent advances provide us the capability to drop-in and seamlessly integrate many types of losses into existing methods, for novel applications.
When one seeks to operate on more than two distributions, however, newer constructions are necessary to effectively estimate distances and transports.
To this end, a well studied idea in the literature is the ``barycenter,''
identified by minimizing the pairwise distance between itself and all other distributions given (analagous to the Karcher mean on Riemannian manifolds). 
The $d$-dimensional proxy distance is then defined as the sum of the distances to the barycenter.

%\begin{comment}
%\vikas{the following paragraph is low on content. Perhaps take 1-2 initial lines, compress and use it close off the previous paragraph?}
%The driving practical focus of optimal transport aims to estimate and minimize the distance between distributions of interest. Within machine learning, this takes the form of a loss; recent developments have enabled constructions of such a loss that allow almost seamless integration into existing methods. These incorporations have led to a number of benefits over traditional mean-squared error approaches, taking direct advantage of the statistical and distributional assumptions and requirements of the model fitter.
%\end{comment}

% {\bf Barycenters.} 
%While distances are defined between two points (or two probability distributions), barycenters extend the idea and enable analyzing many points (or probability 
% One way to quantify dissimilarity between many distributions is through distance to the mean. Here, for averaging,  
% we measure pairwise distances.
% Practically, this has led to models that can 
% concurrently enforce distributional similarity between a 
% set of distributions and has found use in 
% a spectrum of applications, which we will list shortly.

\paragraph{Computing barycenters.}
Assuming that a suitably regularized form of the optimal transport loss is utilized, the pairwise distance 
calculation itself can be efficient -- in fact, 
in some cases, Sinkhorn iterations can be used \citep{cuturi2013sinkhorn}. 
On the other hand, to minimize distances to the mean, 
most algorithms typically operate 
by repeatedly estimating the barycenter and those pairwise distances, and using a ``coupling'' strategy 
to push points toward the barycenter,
or in other cases, summing over all pairwise 
distances. 
% \glenn{I wonder if our method could be also more robust when the pairwise distances are very non-uniform, especially at early iterations...see  \href{https://www.stat.cmu.edu/~larry/=sml/Opt.pdf}  {https://www.stat.cmu.edu/~larry/=sml/Opt.pdf}at the end of page 10}\ronak{I like this idea, maybe we can allude to it briefly?}
%The idea is sensible but a
As the number of distributions 
grows, robustness issues can exacerbate \citep{alvarez2008trimmed} and the procedure becomes
expensive (e.g., for 50 distributions, with 50 ``bins'').
%{\color{red}For example, even on a high end workstation, 
%simply computing the barycenter over
%50 distributions with 50 bins}
%does not converge within a reasonable number of iterations
%using newer packaged solvers. {\color{red} maybe instead of ``not...reasonable'', try ``not competitive"}
%is a time-intensive process.

% {\bf An efficient alternative.}
\paragraph{A potential alternative.}
Multi-marginal optimal transport (MMOT) is a related problem to the aforementioned task but to some extent, the literature has developed in parallel.
In particular, MMOT focuses on identifying a joint distribution such that the marginals are defined by the input distributions over which we wish to measure the dissimilarity.
The definition naturally extends the two-dimensional formulation, and recent work has explored a number of applications \citep{pass2015multi}.
But the MMOT computation can be quite difficult,
and only very recently have practical algorithms been identified \citep{mmotcuturi}.
Additionally, even if a suitable method for computing an analogous measure of distance were available, 
\textit{minimizing} this distance to reduce 
dissimilarity (push distributions closer to each other) is practically hard if standard interior point solvers are needed just to compute the distance itself.
% however very recent research has shown that there exist polynomial time algorithms for this problem 
%have been shown to exist within the last year
% \cite{altschuler2021wasserstein}. Multimarginal versions of optimal transport have been studied but only few have led to practical algorithms \cite{mmotcuturi}.
% \vikas{can we concretize it 
% a little more? cite some numbers on a standard workstation? memory requirements? other resource needs?}
 %{\color{red} Glenn: fairness lead}

% {\color{red}this para does not add much}
% Importantly, we should also appreciate that the calculation of the barycenter is often a means to an end in most learning applications -- 
% the ultimate goal tends to be pushing these distributions to come closer. 
% A barycenter is a modeling choice, but other options may also be viable. 
% Indeed, if a different global measure of ``distributional disparity''
% were to be defined and minimized, the costs associated with pairwise comparisons and optimization
% may potentially be avoided. This hypothesis drives much of our development.

% \vikas{We need some work to justify the role/need of the following paragraph. What is the information is is designed to convey?} \glenn{I think we need to enforce this paragraph by contrasting it with our proposed approach, which I think its the intention of providing this information}
\paragraph{Why and where is dissimilarity important?}
% {\color{red} START} Putting aside the cost to compute the barycenter for the moment, 
% instantiations of such applications often desire to enforce distribution similarity
% only on model outputs.
% For example, in \cite{jiang2020wasserstein}, the authors define fairness measures over the probability of the prediction given ground truth labels.  
% This choice is not without good reasons: discrete outputs lead to extremely nice distributions over the probability simplex, where optimization is easy and assumptions need not be strong.{\color{red} END; could START--END be replaced with the gray text?}
Enforcing distributions to be similar is a generic goal whenever one wishes some outcome of interest to be agnostic about particular groups within the input data.
In applications where training deep neural network models is needed,
% In applications,
it is often a goal to enforce distribution similarity on model outputs. 
For example, in \cite{jiang2020wasserstein}, the authors define fairness measures over the probability of the prediction, given ground truth labels.   
%FIXME: if you read this and know what X Y and Z are, then fill in and uncomment. This leads to distributions over the probability simplex with the desirable qualities X, Y and Z.
% 
However, these methods are rarely extended to continuous measures 
among internal neural network activations,
%(e.g., Bayesian networks), 
mainly due to the strong distributional assumptions needed (product of Gaussians) and the added algorithmic complexity of estimating the barycenter.
These issues limit application of these ideas 
%assumption-free distribution enforcement 
to only the final outputs of neural network models, where the distribution is typically binomial or multinomial.
% One drawback of this choice means that the distributional closeness is only guaranteed over the global model output {\vikas: this is unclear}, and little can be said about layer outputs prior to the thresholding for prediction.
MMOT solutions might be employed here, but suffer similar computational limitations.
% Additionally, full retraining would be necessary if thresholds must be adjusted due to changing business or regulatory requirements.

\paragraph{Contributions.}  \textbf{(1)} We identify a particular form of the discrete multi-marginal optimal transport problem
which admits an extremely fast and numerically robust solution.
Exploiting a recent extension of 
the classical Earth Movers Distance (EMD) to a higher-dimensional Earth Mover's objective,
we show that such a construction is equivalent
to the discrete MMOT problem with Monge costs.
\textbf{(2)} We show that minimization of this \textit{global} distributional measure
leads to the 
harmonization of input distributions very similar in spirit to the minimization of distributions to barycenters (see Figure~\ref{fig:hists}).
\textbf{(3)} We prove theoretical properties of our scheme, and show that 
the gradient can be read directly off from a primal/dual algorithm,
alleviating the need for 
computationally intense 
pairwise couplings needed for barycenter approaches.
\textbf{(4)} The direct availability of the gradient
enables a specific neural network instantiation,
and
with a particular scaffolding provided by differentiable histograms, 
we can operate directly on network activations (anywhere in the network) to compute/minimize the d-MMOT. 
%The proposal integrates nicely with 
%existing pipelines. 
We establish via experiments that computing gradients used in backpropagation is fast, due to rapid access to solutions of the dual linear program.
We compare with barycenter-like approaches in several settings, including common fairness applications.
%, and show that our proposal 
%Our final construction,
%the $d$-dimensional Earth Mover's Distance (DEMD),
%integrates nicely with existing neural network pipelines. 

\begin{figure}[t]
    \centering
    \includegraphics[width=0.19\textwidth]{6_demd/figs/hists/hists_iter_0.pdf}
    \includegraphics[width=0.19\textwidth]{6_demd/figs/hists/hists_iter_100.pdf}
    \includegraphics[width=0.19\textwidth]{6_demd/figs/hists/hists_iter_200.pdf}
    \includegraphics[width=0.19\textwidth]{6_demd/figs/hists/hists_iter_500.pdf}
    \includegraphics[width=0.19\textwidth]{6_demd/figs/hists/hists_iter_1000.pdf}
    \caption[Minimizing the generalized Earth Mover's Distance]{Starting and ending state of minimizing a multi-marginal OT distance. Each iteration minimizes the generalized Earth Mover's objective, and then updates each histogram in the direction provided by the gradient.}
    \label{fig:hists}
\end{figure}

%%%% OLD
% In this work, we identify a simple {\em global} distributional measure with scaffolding that allows for fast computation over any continuous model output. A new multi-dimensional generalization of the classical Earth Mover's distance has recently been shown to be efficiently computable, and further developments show  that its minimization is extremely fast compared to existing barycenter-style methods.
% Functionally, it
% possesses several useful ingredients: (a) it requires no distributional assumptions, (b) it is efficient to compute, (c) its gradient is also efficiently computable. As a result, the functional can be 
% directly used in regimes where gradients are required for optimization (e.g., backpropogation and other gradient descent methods). 
% Finally, the functional is  numerically stable, and has an intuitive interpretation which facilitates interpretative reasoning.  
% In this work, we identify a simple {\em global} distributional measure with scaffolding that allows for fast computation over any continuous model output. A new multi-dimensional generalization of the classical Earth Mover's distance has recently been shown to be efficiently computable, and further developments show  that its minimization is extremely fast compared to existing barycenter-style methods. Functionally, it
% %distributions are computed over the invariant sets, a barycenter is computed as for the measure of invariance, and heuristic methods are employed to reducing the measured invariance in the loop. 
% Although these approaches aim to directly address the heterogeneity in the model output, they require (1) particular algorithms and solvers for relaxing discrete outputs to enable backpropogation and (2) that operating points and thresholds determining the final prediction are fixed. 
% If a method to operate directly on the activations prior to thresholding was feasible...
% Operating directly on the activations has its own hurdles.
% The activations are continuous, and so distributional assumptions must be made if any computation in the learning pipeline is to remain tractable. However a priori these assumptions are explicitly unknown. 
% \vikas{this appears to be tailored to the above paragraph. Ideally, we should start by highlighting a feature that allows moving away from the pairwise iterative scheme of barycenters.}
% In this work, we observe that a \textit{discretization} of continuous layer outputs allows for the computation of a new multi-distributional generalization of the classical Earth Mover's Distance (EMD) that allows for drop-in computation of disparity at the activation-level. 
% A common application of optimal transport incorporates the concept into a machine learning model by appending a suitable regularization term to the model's objective function.  Typically, this regularization term has the effect of making the statistical distribution of subsets of solutions either more uniform or else invariant in some prescribed way.  An illustration of a prototypical process appears in Figure~\ref{fig:hists}.
% In this paper, we apply a multi-distributional generalization of the classical Earth Mover's distance to the task of regularizing network models.
% The generalization that we employ is based on the classical Earth Mover's Distance (EMD), and
% We demonstrate that this functionally 
% possesses several useful ingredients: (a) it requires no distributional assumptions, (b) it is efficient to compute, (c) its gradient is also efficiently computable. As a result, the functional can be 
% directly used in regimes where gradients are required for optimization (e.g., backpropogation and other gradient descent methods). 
% Finally, the functional is  numerically stable, 
% and has an intuitive interpretation which facilitates interpretative reasoning.  
% Empirical demonstrations of the speed and viability of our approach are presented in Section~\ref{sec:results}.
% We take direct advantage of linear programming formulations of the EMD in its generalized form, and show that in fact the gradient can be directly identified as the solution to the dual program. Recent algorithmic developments allow for the primal and dual to be solved concurrently, allowing for a no-cost gradient computation if the forward computation is chosen accordingly.
% Whereas the classical Earth Mover's distance defines a distance metric between pairs of distributions, the generalized Earth Mover's objective defines a natural way to  measure  dissimilarity amongst $d\geq 2$ distributions. 
% Both the classical distance and the generalized objective can be expressed as linear programs.
% One advantage of the linear program formulation is the fact that every linear program has an equivalent {dual} linear program.
% We demonstrate, using standard sensitivity analysis, that solutions to the dual linear program equal the primal objective functional's gradient. If the functional is regularizing a neural network model, this gradient can be used for backpropagation of derivatives during model training.
% Additionally, it was recently shown that solving both the primal and dual linear programs of the generalized Earth Mover's objective can be accomplished in linear time and constant space by applying a straightforward  greedy algorithm. In experiments, we compare the speed to compute the gradients of the generalized Earth Mover's objective of the greedy algorithm against native auto-gradient procedures, and we demonstrate over $1000$-fold speed improvement. The method we describe is numerically stable, as no division operations or unusually large quantities are invoked in the process of solving the linear programs.  
% In many settings, however, a strict set of assumptions is required to apply and compute these existing methods. Particularly, optimal transport measures over continuous spaces are only practical when strong distributional assumptions can be made, and while discrete assumptions allow for more broad computation, they tend to poorly capture and generalize the actual measures of interest.
% An important feature of our approach is that it acts on model outputs prior to thresholding, so invariances to be maintained across varying thresholds.
% Our procedure addresses both of these limitations directly, by first promoting invariance at all discretizations of model outputs \textit{prior to thresholding}, and by making use of a differentiable histogramming procedure to directly apply precomputed gradients through backpropagation.
% A challenge in practical implementations that use this architecture  define invariance can be based on business or regulatory constraints, but these constraints can change over time.  Industry deployments of ML pipelines are not compatible with such a setting is an important way: thresholds considered for discretization or classification are often adjusted or changed due to changing business or regulatory directives.   After training with procedures similar to the above, no guarantee regarding invariance can be made if the threshold is changed. It is often not sufficient or even feasible to retrain when operating points may or must change.
% {\color{red} redundant, merge with above para}
% \noindent\textbf{Contributions.} In this work, we present an application of a recent extension of the classical Earth Movers distance to a higher-dimensional Earth Mover's objective.
% We show that minimization of the objective leads to the harmonizing (lang) of input distributions similar to the minimization of distributions to barycenters.
% We prove theoretical properties of the objective and the procedure that
% reveals the gradient can be read directly off from a primal/dual algorithm,
% alleviating the need for computing intensive pairwise couplings.
% % This objective provides an alternative to barycenter approaches as a way to regularize network models in a way that reduces statistical dissimilarity of the optimal solutions. % The functional we use acts on a \textit{discretized version} of model outputs prior to thresholding, and allows for invariances to be maintained across varying thresholds. 
% With a particular instantiation of differentiable histograms, we can apply and smoothly operate directly on network activations to compute the EMD measure. 
% We establish through experiment that the speed of computing gradients used in backpropogation can be computed in substantially shorter times than one can achieve using standard tools, due to rapid access to solutions of the the dual linear program formulation.
% We compare and contrast the performance and speed of our construction against Barycenter-like measures in a number of settings and demonstrate applications in a common fairness setting.
% Our final construction integrates seamlessly with existing neural network pipelines.
% Machine learning has become ubiquitous, but has suffered some setbacks in public and regulatory visions due to issues typically considered orthogonal to traditional measures of success within the literature.
% Most modern advancements in machine learning follow from learning a model that optimizes for some measure of \textit{accuracy}, measured by misclassification errors over all samples or global measures of true and false positive and negative rates.
% Unfortunately, when statistical assumptions of homogeneous data fail, these measures can be skewed heavily towards subsets of the data that follow a majority or the primary mode of the underlying distribution. These biases can take many forms, and in applications in which data and outcomes may be represent people and decisions for those people, can lead directly towards unfairness and disparity among groups.
% The last few years of machine learning deployment has revealed a myriad of situations in which strong bias and fairness concerns have led to strong social backlash. Examples: COMPAS, twitter, etc. \cite{?}
% With these concerns has also come interest in technical solutions to fairness and bias. Several conferences and workshops have been created \cite{fat, facct, etc}, devoted to grounding and understanding these ideas within a mathematical and statistical context. From these communities concrete definitions have been formed, along with a number of algorithmic solutions to both avoid and account for bias and unfairness within both data and models.
% Existing methods for accounting for fairness typically define measures over discrete output spaces, and attempt to minimize some measure of disparity between these discrete probabilities. Varying definitions of fairness have been constructed under this umbrella, and a number of nice solutions have been proposed that allow for fair regularization in traditional machine learning training pipelines.
% A drawback to these approaches is that they rely on the discrete/binary outputs of models to be fixed: once a model is trained, the threshold used for those outputs cannot be changed else all fairness guarantees provided by the regularization fail, because the output distributions have all been constructed with that specific threshold. Practical deployments of machine learning models in industry rely on being able to select different points of operation, to match secondary requirements or measures provided by business or regulatory requirements. In these cases it is not sufficient or feasible, especially for large models, to create and retrain methods when these operating points must change. Their is a clear need for methods which are fair across potentially many points of operation.
% While moving to fairness over the continuous measure prior to thresholding may be infeasible, new results suggest that discretization at points of operation can lead to practical solutions for constructing models fair over many thresholds.
% \noindent\textbf{Contributions.} In this work, we analyze fairness from a new perspective. By looking at the distances between distributions over discretized spaces prior to thresholding for downstream discrete tasks, we apply a recently developed tool for computing EMD distance to quickly measure the disparity between groups over discretized continuous measures. Using a key observation that the gradients are readily available, we directly incorporate this EMD distance into standard machine learning pipelines. 
\section{Related Work}
%While machine unlearning has been studied by many in the field, to the best of our knowledge we are the first to propose 
To contextualize our contributions, 
we briefly review existing proposals for machine unlearning. 

\noindent\textbf{Na\"ive, Exact Unlearning.}
A number of authors have proposed methods for exact unlearning, in the case where $(\epsilon=0, \delta=0)$. SVMs by \cite{romero2007incremental,karasuyama2009multiple}, Na\"ive Bayes Classifiers by \cite{cao2015towards}, and $k$-means methods by \cite{ginart2019making} have all been studied. 
%More recently, \cite{} develop methods for Random Forests.
But these algorithms do not translate to stochastic models with millions of parameters.

\noindent\textbf{Approximate Unlearning.} 
With links to fields such as robustness and privacy, we see more developments in approximate unlearning under Definition~\ref{def:forget}. 
The so-called $\epsilon$-certified removal by \cite{guo2019certified} puts forth similar procedures when $\delta=0$, and the model has been trained in a specific manner.
\cite{guo2019certified,izzo2020approximate} provide updates to linear models and the last layers of networks, and 
\cite{golatkar2020forgetting,golatkar2020eternal} provide updates based on linearizations that work over the full network, and follow-up work by \cite{Golatkar_2021_CVPR} presents a scheme to unlearn under an assumption that some samples will not need to be removed.

Other recent work has taken alternative views of unlearning, which do not require/operate under probabilistic frameworks, see \cite{bourtoule2021machine,neel2021descent}. These schemes present good guarantees in the absolute privacy setting, but they require more changes to  pipelines (sharding/aggregating weaker models) and scale unsatisfactorily in large deep learning settings.
\section{Background}
% Before stating the generalized Earth Mover's problem, it is helpful to have a complete description of the classical program.  No new concepts are required to understand the general program once the classical program has been grasped. 
% \noindent\textbf{Notations.}

Denote by $[n]\coloneqq\left\{1,\ldots,n\right\}$, the set of positive integers no larger than $n$. For elements $x\in\RR^{n}$, we denote the $i$th entry of $x$ as $x(i)$, e.g.,  
for any $x\in\RR^{n}$, $x=(x(i):i\in[n])$.    
The positive orthant of $\RR^n$ is denoted $\RR_{+}^{n} \coloneqq \left\{x\in\RR^{n}:x(i)\geq 0, i\in [n]\right\}$. 
We denote by $e\coloneqq (1,\ldots,1)\in\RR^{n}$ the constant vector. 
For $q\geq 1$, we define the $q$-norm  as $\norm{x}_q\coloneqq \left(\sum_{i\in[n]}\left|{x(i)}\right|^{q}\right)^{1/q}$, and if $q$ is suppressed, then $\left\|x\right\|\coloneqq \left\|x\right\|_2$. 
A discrete probability distribution is a point $p\in\RR^{n}_+$ with $e'p=\norm{p}_1=1$.

Given a pair of discrete probability distributions $p_1,p_2\in\RR^{n}_+$, we may want to quantify similarity or dissimilarity. 
%$p_1$ and $p_2$ are. 
Often we do this by selecting from many measures, including the $q$-norm, KL-divergence or the Earth Mover's Distance (EMD).
%
% \vikas{describe that we want to find how similar $p_1$ and $p_2$ are}
%This problem is most well known as discrete optimal transport, or the Earth Mover's problem, and can be described as a linear program.
The EMD for a pair of distributions has several equivalent interpretations. First, let $p_1$ be a source of mass, and $p_2$ be a sink for mass, and  $x(i,j)$, where  $x\in\RR^{n\times n}$, represent the flow of mass from $p_1(i)$ to $p_2(j)$.
Denote by $c(i,j)$ the cost of moving one unit of mass from  $p_1(i)$ to $p_2(j)$.
The EMD between $p_1$ and $p_2$ is the minimal cost to transform $p_1$ into $p_2$, 
%given by the sum of the costs associated with shifting mass, according to $x(i,j)$ such that $p_1$ is transformed into $p_2$.
% Unless otherwise specified, we assume $c(i,j)=|i-j|$, which corresponds to ground distance. 
%This can be 
written as a linear program (LP):
\begin{align}\label{eq:2demd}
\begin{aligned}
\underset{{x\in \RR^{n\times n}_+}}{\textrm{min}} \sum_{i,j} c(i,j) x(i,j) \quad  \textrm{s.t.}\quad \sum_j x(i,j) &= p_1(i); \ 
\sum_i x(i,j) = p_2(j),\ (\forall i,j\in[n]).
\end{aligned}
\end{align}
The source-sink interpretation is asymmetric in $p_1$ and $p_2$, but the LP is symmetric in $p_1$ and $p_2$.  It can be shown that the objective value of this LP defines a {\em metric} \citep{kantorovich1960mathematical}, and the optimal value of the objective function can be interpreted as a distance between $p_1$ and $p_2$,  %For this reason, the Earth Mover's Distance is 
and useful to quantify dissimilarity between pairs of distributions. In particular,  $p_1=p_2$ if and only if the optimal objective value of the Earth Mover's problem vanishes.
%
%Since the Earth Mover's program is a linear program, it has 
The LP in (\ref{eq:2demd}) has an equivalent dual LP, 
%which has the form
\begin{align}\label{eq:2dualemd}\begin{aligned}
    &\underset{z_1,z_2\in\RR^{n}}{\textrm{max}} z_1'p_1 + z_2'p_2 \quad 
    \textrm{s.t.}\quad  z_1(i) + z_2(j)\leq c(i,j),\  (\forall i,j\in [n]).
    \end{aligned}
\end{align}
By strong duality, the optimal value of the primal program~(\ref{eq:2demd}) equals the optimal value of the dual program (\ref{eq:2dualemd}). 
% We show below that a solution to dual program is useful for computing gradients.
Many practical relaxations have been proposed for (\ref{eq:2demd}), including entropic regularization \citep{cuturi2013sinkhorn}.
%, which is notable for being both practical and scalable.
%as it has led to practical approximations over large histograms.
Computation of the EMD is readily available as in the Python Optimal Transport (POT) library \citep{flamary2021pot}.

%Minimization of the EMD is feasible via backend-supported implementations such as POT \citep{flamary2021pot}.


% This can equivalently been seen as identification of a joint distribution $x$ with both $p_1$ and $p_2$ as marginals, where the total energy of the joint $x$ is minimized under some cost $c$. 
% Other interpretations include the flow of mass between a source and a sink.




% Outcome variable $Y$, discrete random. $y \in \{0,1\}$.
% Sensitive Attribute $A$, random discrete variable. $a \in A$.

% A sample may be a tuple of $\{x_i, a_i, y_i\}$, and a model may take in the subset $\{x_i, a_i\}$ and generate a prediction $\hat{y}_i = f_\theta(x_i, a_i)$. Dataset $S:= \{x_i,a_i,y_i\}_{i=1}^n$. A fairness measure $G(a,y,\hat{y})$.

% Demographic Parity
% \begin{align}
%     G(a,y,\hat{y}) &= p(\hat{Y} = \hat{y} | A = a) \\
%     g(S, a, y, \hat{y}) &= \frac{1}{n} \sum_{i=1}^n \II[a_i = a, \hat{y}_i = y] \\
%     &= p_a
% \end{align}
% $Y$ binary, so we don't need to distinguish $p_a$ for different $y$.

% Fairness across all groups $\forall a \in A$ requires finding a point (model, parameters, $\theta$) where these probabilities are close.
% \begin{align}
%     \theta^* = \min_{\theta} \sum_{i,j} d(p_{a_i},p_{a_j})
% \end{align}
% For each group, there may be an optimal $\theta_a$. Then we can rewrite the problem as
% \begin{align}
%     \theta^* = \min_{\theta} \sum_a d(\theta,\theta_a)
% \end{align}
% General formulation \cite{jiang2020wasserstein} requires computing the wasserstein-1 barycenter, computationally intensive, many approximations, complicated algorithm, requires entropic regularization, sinkhorn iterations.

% Many calls to 1-d wasserstein problem.
% \begin{align}
%     d(\theta_1, \theta_2) = &\min \langle P, C \rangle \\
%     &s.t. \sum_j P = \theta_1,\ \sum_i P = \theta_2
% \end{align}
% For fairness with two groups, the distributions can be the conditional fairness definitions (DP, EO, etc.).

% How about with d groups? d-dimensional tensor.
% \begin{align}
%     \min \langle P, C \rangle_{\otimes}
% \end{align}
% How do we solve this efficiently? \cite{kline2019properties}
% Kline 2019: d-dimensional Earth Movers.

% Main technical piece: Using new solutions to the d-dimensional EMD for barycenter computation.

% d-dimensional earthmover's problem.
% \begin{align}\label{eq:dEMD}
%     &\min \sum_{I \in \cI} C_I P_I \\
%     &s.t. \sum_{I\setminus j} P_{I\setminus j} = x_j \quad \forall j \in \{1,\ldots, d\}
% \end{align}

% \subsection{Subgroup Fairness}

% We may not only have one sensitive attribute $A$. Let $\cA$ be a collection of binary attributes. In this case, if we would like to be fair for all groups, simply enforcing the above definitions for each groups would work, but may not necessarily lead to fairness for various \textit{intersections} of subgroups. In this case, what we may be interested in is subgroup fairness. 

% Following traditional definitions, let $G \in \cG$ be a specific subgroup for which we wish to be fair to, identified by a binary variable defined as the intersection over the $A$'s for each larger group. For example, we may have $A_1$ as race and $A_2$ as gender, and we may wish to be fair towards an intersection $G := A_1 \cap A_2$.

% This definition also allows for one-hot encodings of categorical variables, and intersections of different categorical variables.

% It is easy to see that with many different groups and subgroups we may wish to be fair to, the total set size of $\cG$ grows exponentially with the number of sensitive attributes. 

% Following a traditional approach of optimizing each group towards a globally fair model is not feasible for existing Wasserstein-style approaches.

% Entropic regularization approaches with Sinkhorn iterations grow with the number of distributions, and computing Wasserstein barycenters for many groups via the above formulations requires computing barycenters over all subgroups and estimation of these centers become worse with more samples. (See below for speed comparisons)

\subsection{Discrete Multi-Marginal Optimal Transport}

%The natural next step from above is an extension when we have an arbitrary number of $d>2$ distributions. 
%A limitation of t
The foregoing approach 
%is that it 
applies only to $d=2$ distributions, namely $p_1$ and $p_2$. We briefly review the extension 
%of the above idea 
to $d>2$ distributions; 
%. The optimal transport 
the literature calls this \textit{multi-marginal optimal transport (MMOT)}.%, with similar subproblems following problem and cost-specific assumptions a la Kantorovich and Monge. Here we focus on the discrete setting.
% \vikas{give the reader a sense of whether this is still background material and/or where this is going.}

\begin{definition}[Discrete Multi-Marginal Optimal Transport (d-MMOT)]\label{def:dmmot}
Let $p_1, \ldots, p_d\in\RR^n_{+}$ be discrete probability distributions. %Then, $e'p_i=1$ for all $i\in[d]$. % with each $p_i \in \Delta^n$, where $\Delta^n$ is the $n$-dimensional simplex. 
%Let $c_d : \cX^d \rightarrow \RR_{\ge 0}$. 
Let $C_d : \RR^{n^{d}}\rightarrow \RR_{+}$. 
The discrete multi-marginal optimal transport problem (d-MMOT) can be written as
\begin{align*}%\label{eq:dmmot}
    %&\underset{{X \in \cX^d}}{\textrm{minimize}}\quad C_d(X) \\
    &\underset{{X \in \RR^{n\times \cdots \times n}}}{\textrm{min}}\quad C_d(X) \quad \textrm{s.t.}\quad X_i = p_i,\ (\forall i\in [d]),
\end{align*}
where $X_i \in \RR^n$ is the $i$-th marginal of  $X \in \RR^{n\times \cdots \times n}=\RR^{n^{d}}$.
\end{definition}
Following the original formulation \citep{kantorovich1942}, we will restrict the cost function $C_d(\cdot)$ to the linear map, $C_d(X) \coloneqq \langle c, X \rangle_{\otimes}$, where $c \in \RR_{+}^{n\times \cdots \times n}$ is nonnegative.
%The sequel will focus on the following form of the d-MMOT problem.
Here, the d-MMOT is the LP,
\begin{align}\label{eq:gemd}\begin{aligned}
    \underset{x\in\RR^{n^{d}}_{+}} {\textrm{min}}
    \sum_{i_1,\ldots,i_d} c(i_1,\ldots, i_d)\, x(i_1,\ldots,i_d) \quad \textrm{s.t.}
    \sum_{i_2,\ldots,i_d} x(i_1,\ldots,i_d) &= p_1(i_i), (\forall i_1\in[n])\\
    % &\sum_{i_1,i_3\ldots,i_d} c(i_1,\ldots, i_d)x(i_1,\ldots,i_d) = p_2(i_2),\ (\forall i_2\in[n])\\
    \qquad\vdots\\
    \sum_{i_1,\ldots,i_{d-1}} x(i_1,\ldots,i_d) &= p_{d}(i_{d}), (\forall i_d\in[n]).
    \end{aligned}
\end{align}
This linear program (LP) is central to the regularization schemes that are discussed below.
\section{Efficient d-MMOT Computation}
The linear d-MMOT problem (\ref{eq:gemd}) suffers from the curse of dimensionality: the LP has $n^d$ variables, and even modest choices of $n$ and $d$ can result in a LP with billions of variables, making standard LP solvers inapplicable.
%Traditional linear programming solvers such as interior point methods have been studied in the two-dimensional context, but the MMOT setting does not allow a direct extension given the exponential dependence on $d$. 
Alternatively, specific algorithms have been proposed  \citep{benamou2015iterative}, and 
relaxations via entropic regularization have become more widespread, with very recent extensions to the d-MMOT setting \citep{tupitsa,mmotcuturi}. 

% Putting aside the computation of the dMMOT objective,
% in practical applications its often the case that moving input distributions closer or further from each other requires minimizing or maximizing the objective. Define an arbitrary distance among distributions as $\phi(p_1, \ldots, p_d)$. If this distance is defined as the objective in \eqref{eq:gemd},
% then minimizing the distance among distributions in this form requires bilevel optimization.
% Of course we can set all distributions to be the same for the problem to minimized,


In practice, the cost $c$ in (\ref{eq:gemd}) typically takes one of two forms. In the case where the distributions $p_1, \ldots, p_d$ are over categorical variables, the cost is typically defined as $c(i_1, \ldots, i_d) = 0$ when $i_1 = \cdots = i_d$ and $1$ otherwise.
However, and importantly, if the distributions $p_i$ are histograms over some ordinal or discretized space, 
the cost typically has a structure 
closer to that of a ``tensorized'' distance, characterized by the {\em Monge} property.
%\ronak{emphasize this, it is key to distinguish our construction, maybe add motivation in intro re this}
\begin{definition}[Monge Property]
A tensor $c$ is Monge if for all valid $i_1, \ldots i_d$ and $j_1, \ldots, j_d$, 
%$1 \leq i_1 \leq j_1 \leq n$, $1 \leq i_2 \leq j_2 \leq n$, $1 \leq i_d \leq j_d \leq n$,
\begin{align}
c(s_1, \ldots, s_d) + c(t_1, \ldots t_d) \leq c(i_1, \ldots i_d) + c(j_1, \ldots, j_d)
\end{align}
where $s_k = \min(i_k, j_k)$ and $t_k = \max(i_k, j_k)$.
\end{definition}
Our focus is on a specific cost, which is known to be Monge: $c(i_1,i_2,\ldots,i_d)\coloneqq \max{\{i_k:k\in[d]\}} - \min{\{i_k:k\in[d]\}}$. When $d=2$, this cost reduces to $c(i_1,i_2)=|i_1-i_2|$, which agrees with the classical EMD cost. This choice of $c$ is called the {\em generalized EMD cost}.


%\ronak{this is not explained below, a sentence here in place suggesting how its used in histograms/etc would  be better}

% While the Monge restriction may seem ``nicer" compared to arbitrary nonnegative costs, it is not yet clear how this can be leveraged for efficient computation. 
% The Monge cost does not in itself provide nicer properties to existing methods for multi-marginal optimal transport,
% the program suffers from the curse of dimensionality.
% If $d$ histograms with $n$ bins are being compared, the decision variable of the program still lies in $\RR^{n^{d}}$, which can be very large even if both $d$ and $n$ are modest (say $n=d=10$). 
% To progress in any direction, it is necessary for us to identify a method for taking advantage of this structure in an efficient manner.

%\vikas{the following needs attention. We should think of how to provide some daylight -- because as written, it seems what a reader may call a ``direct application''.}
\begin{remark}Limiting our attention to this cost is not as restrictive as it may appear. 
Indeed, \cite{BEIN199597} shows that the optimal solution to the LP (\ref{eq:gemd}) is independent of the cost, as long as it is Monge.
Additionally, when $c$ is the generalized EMD cost, \cite{kline2019properties} describes a greedy algorithm that solves both (\ref{eq:gemd}) and its dual (\ref{eq:dualgeneralemd}) in linear time. 
% They present an algorithm (presented in Algorithm~\ref{alg:primaldual})
% and constructs a theoretical and algorithmic framework for solving the discrete MMOT.
% Particularly,
% \ronak{key results from kline19, complexity, short description of results}
%\begin{remark}
It is also shown that the optimal objective value is a continuous nonnegative function of each probability distribution $p_j$ (i.e., small changes in one distribution cause small changes in the objective value). Continuity is critical for numerical stability.  Next, when $c$ is the generalized Earth Mover's array, the optimal objective value vanishes if and only if $p_i=p_j$ for all $i,j\in[n]$. 
%\end{remark}
This {\em separability} property is useful in applications where we wish to iteratively ``step towards'' the barycenter of a set of distributions, {\color{blue}see Fig. \ref{fig:min_demd}}.
%{\color{blue} This typically would happen in an {\em outer} loop when training a deep neural network. }
In order to step towards {\color{blue}(red arrows)} a barycenter, we  require a descent direction.  %This is the next result.
%While we can directly push them all towards an arbitrary point in the simplex (e.g., uniform, barycenter), it would be better if we could naturally move towards a point that is close, in some sense, to the initial distributions.
% is relevant since our goal is to regularize the model in a way that favors equi-distribution of mass across multiple probability densities.
We observe that the following  result gives us  this functionality.
\end{remark}

% We can also write the dual linear program.
% \begin{alignat}{2}\label{eq:dualgeneralemd}\begin{split}
% &\underset{z_j\in\RR^n, j\in[d]}{\textrm{maximize}}\qquad\sum_{j} x_j'z_j\\
% &\textrm{subject to}\qquad z_{1}(i_1)+\cdots+z_{d}(i_{d})\leq c(i_1,\ldots,i_{d}),\end{split}
% \end{alignat}
% where the indices in the constraints include all $i_j\in[n]$, $j\in[d]$.

% The following result will make use of the dual linear program.  Before stating the theorem, we introduce some notations. Given distributions $p_1,p_2,\ldots,p_d\in\RR^{n}$ with $e'p_j=1$ for all $j\in[d]$, denote by $\phi(p_1,\ldots,p_d)$, the optimal objective value of the LP in (\ref{eqn:gemd}). Denote by
% \begin{align*}z^*=(z_1^{*}, z^*_2,\cdots,z^*_{d}),
% \end{align*}
% an optimal solution to the dual program~(\ref{eq:dualgeneralemd}).
\begin{theorem}
The dual linear program of the d-MMOT problem (\ref{eq:gemd}) is
\begin{alignat}{2}\label{eq:dualgeneralemd}\begin{split}
\underset{z_j\in\RR^n, j\in[d]}{\textrm{maximize}}\qquad\sum_{j} p_j'z_j 
\qquad \textrm{subject to}\qquad z_{1}(i_1)+\cdots+z_{d}(i_{d})\leq c(i_1,\ldots,i_{d}),\end{split}
\end{alignat}
where the indices in the constraints include all $i_j\in[n]$, $j\in[d]$.
Denote by $\phi(p_1,\ldots,p_d)$, the optimal objective value of the LP in (\ref{eq:gemd}). Let $z^*$ be an optimal solution to the dual program~(\ref{eq:dualgeneralemd}).
Then,
% In the established notation, the following two claims hold. First,
\begin{align*}
\nabla \phi(p_1,\ldots,p_{d}) = z^*, 
%\end{align*}
~~\text{and for any $t\in \RR$,}~~
%\begin{align*}
\phi(p_1,p_2,\ldots,p_{d}) = \sum_{j}p_j'
(z_j^* + t\, \eta),
\end{align*}
where $\eta\coloneqq (z_1^{*}(n)\,e, z^*_1(n)\,e, \cdots, z^*_{d}(n)\,e)$.
\label{thm:dualgrad}
\end{theorem}
\begin{proof}
The main observation invokes perturbation analysis \citep{mangasarian1979nonlinear,ferris1991finite} of LPs to assert that, under mild uniqueness conditions, small changes to a LP's input data does not change its optimal solution. The full proof is in Appendix~\ref{sec:app-proof}.
\end{proof}
\begin{remark} The first part of this result provides what we require: a direction of descent. 
Thus, if we can solve the d-MMOT problem and also find the optimal solution to its dual, 
then we can step (or move) our distributions in the opposite direction of the dual variables
to push them together, {\color{blue}see Fig. \ref{fig:min_demd}}.
The second claim is somewhat technical, and reconciles particular affine shifts that result in equivalent objective values.
\end{remark}

% The second claim, which is rather technical, requires some motivation/explanation for why it was included. 
% We observed empirically that solutions yielded by either autograd or by numerical approximations differed from the solutions yielded by the greedy algorithm that we employ. 
% But the difference was, in all cases, inconsquential: the dual objective functional, evaluated on both vectors agreed, and in all examples tested, the difference equaled a constant factor times the vector $\eta$ that is stated in the theorem.
{\color{blue} 
\subsection{Optimization of d-MMOT}\label{sec:dual}

{\bf Setup.} 
We can now instantiate d-MMOT as an add-on term in a standard machine learning formulation. Concretely, it can be positioned alongside typical learning losses $L(f(x),y;\theta)$ to encourage minimizing distances among $d$ different distributions $g_i\in G, i \in [d]$, $\text{d-MMOT}(f(x),g;\theta)$, i.e., 
$\min_{\theta} L(f(x),y;\theta) + \text{d-MMOT}(f(x),g;\theta)$. 
Within a deep neural network (DNN) architecture, as in some of our experiments,  
several properties of the d-MMOT module are useful:
our results above naturally provide clean operations for computing both the required objective in the ``forward'' pass and gradients in the ``backward'' pass.
}


\begin{figure*}
    \centering
    %\includegraphics[trim={0 14.5cm 2cm 5cm},clip,width=\textwidth]{6_demd/figs/demd_abstract.png}
    % \includegraphics[width=0.95\textwidth]{6_demd/figs/bary-gem-step.pdf}
    %\includegraphics[width=0.9\textwidth]{6_demd/figs/demd_hull_new.png}
    \includegraphics[trim={2cm, 0, 0, 18cm},clip,width=0.9\textwidth]{6_demd/figs/demd_hull_v3.png}
    \caption{\footnotesize ({\em Left}) Barycenter methods identify a center (red circle) and transport \textit{all} distributions (blue circles) toward that center along the coupling path (yellow). ({\em Right}) Our DEMD approach identifies ``support" distributions that lie on the convex hull (outlined circles),  and only those distributions are moved in a direction that decreases the Generalized EMD objective.}
    \label{fig:min_demd}
    \vspace{-5pt}
\end{figure*}

% \subsection{Solving the dMMOT program and its dual}\label{sec:dual}
% JK: this paragraph interrupts the flow
%Using either the iterative Bregman or Sinkhorn-style methods within an outer optimization is  impractical: invoking an interior point solver at each outer iterate explodes an already exponentially large problem, and methods extending two-dimensional entropic relaxations such as unrolling add significant complexity and require approximate solutions.

% Linear program solvers are now a  mature technology,  with polynomial-time guarantees for solution accuracy.
% Despite this, they are unsuitable as the engine that solves the programs of our intended use-case, namely regularizing a network model. 
% Our intent is to have the ability to embed a process that solves an instance of (\ref{eqn:gemd}) and its dual repeatedly during network training.  One reason standard methods are too slow for this is that the programs themselves can be quite large: recall that the cost functional is a dense, rank-$d$ tensor. We note 
% however that there is recent progress towards differentiable LP modules  \cite{meng2020differentiable,NEURIPS2020_6bb56208}. 


% \begin{remark}
% % While recent progress has been made towards differentiable LP modules \citep{meng2020differentiable,NEURIPS2020_6bb56208},
% % here we instead observe that 
% Indeed, these approaches are a necessary path forward given the problem as written,
% where infeasibility is proven for the unregularized formulation (Theorem 3.3 in \cite{mmotcuturi}).
% Notably, their proof follows directly by identifying a form of the cost tensor $C$ such that a reduction to a matching problem leads to a contradiction in theoretical hardness.
% % \vikas{some form of the following sentence needs to be positioned a bit earlier and perhaps repeated here. We need to contrast it with the literature. Is this novel/new etc.}
% In what follows we identify a specific form of the cost tensor $C$ that not only allow for a tractable algorithm, but also has key properties that can be exploited in downstream applications.
% \end{remark}


% In practice, the form of the cost $C$ typically takes two forms. In the case where the distributions $p_1, \ldots, p_d$ are over categorical variables, the cost is typically defined as $c(i_1, \ldots, i_d) = 0$ when $i_1 = \cdots = i_d$ and $1$ otherwise.
% On the other hand, if the distributions $p_i$ are histograms over some ordinal or discretized space, 
% the cost typically follows a structure 
% closer to that of a metric.
% Cost tensors that form these sorts of metrics can be characterized by a tensor form of a distance. \ronak{emphasize this, it is key to distinguish our construction, maybe add motivation in intro re this}
% \begin{definition}[Monge Array]
% A tensor $C$ is Monge if for all valid $i_1, \ldots i_d$ and $j_1, \ldots, j_d$, 
% %$1 \leq i_1 \leq j_1 \leq n$, $1 \leq i_2 \leq j_2 \leq n$, $1 \leq i_d \leq j_d \leq n$,
% \begin{align}
% c(s_1, \ldots, s_d) + c(t_1, \ldots t_d) \leq c(i_1, \ldots i_d) + c(j_1, \ldots, j_d)
% \end{align}
% where $s_k = \min(i_k, j_k)$ and $t_k = \max(i_k, j_k)$.
% \end{definition}

% While the Monge restriction may seem ``nicer" compared to arbitrary nonnegative costs, it is not yet clear how this can be leveraged for efficient computation. 
% The Monge cost does not in itself provide nicer properties to existing methods for multi-marginal optimal transport,
% the program suffers from the curse of dimensionality.
% If $d$ histograms with $n$ bins are being compared, the decision variable of the program still lies in $\RR^{n^{d}}$, which can be very large even if both $d$ and $n$ are modest (say $n=d=10$). 
% To progress in any direction, it is necessary for us to identify a method for taking advantage of this structure in an efficient manner.

% \vikas{the following needs attention. We should think of how to provide some daylight -- because as written, it seems what a reader may call a ``direct application''.}
% Recent work by
% \cite{kline2019properties} leverages the properties of the dMMOT problem with this cost, albeit with a focus on the specific $d$-dimensional earth mover's construction. Interestingly, the algorithm and analysis therein can be leveraged to solve the d-MMOT problem.
% They present an algorithm (presented in Algorithm~\ref{alg:primaldual})
% and constructs a theoretical and algorithmic framework for solving the discrete MMOT.
% Particularly,
% \ronak{key results from kline19, complexity, short description of results}

{\bf Using the primal and dual variables.}  If the optimal objective value of (\ref{eq:gemd}) serves in regularizing a deep neural network, then we can train the network as follows. %this means can compute the objective value during a the forward pass
%The first issue is trivial to implement within standard neural network frameworks. 
% During the forward pass, one can compute the EMD term using the primal/dual greedy algorithm of \cite{kline2019properties}, with one simple modification:  prior to returning, the dual solution variables are stored for use during the backward pass. 
During the forward pass, i.e., computing the d-MMOT objective, we can employ a version of the aforementioned primal/dual algorithm that solely computes the function $\phi$ and stores the dual variables $z$.
As backpropagation proceeds, when the EMD module encounters an incoming gradient, it is simply multiplied by the stored dual variables (see Theorem~\ref{thm:dualgrad} and Algorithm~\ref{alg:demdFunc}). We call our procedure the $d-$dimensional Earth Mover's Distance, or in short,\textit{ the DEMD algorithm}. 

\iffalse
\begin{algorithm}
\caption{Our proposed method: $d-$Dimensional Earch Mover's Distance (DEMD)}\label{alg:demdFunc}
\begin{algorithmic}
 \Function {Forward}{$p_1, \ldots, p_j$}
    \State Compute $\sum s_k t_k$ and $(z_1,\ldots,z_d)$ via the primal/dual algorithm in \cite{kline2019properties}.% \ref{alg:primaldual}.
    \State save $(z_1,\ldots, z_d)$.
    \State \Return $\sum_k s_k t_k$
 \EndFunction
 \Function {Backward}{$\text{gradOutput}$}
 	\State load $(z_1,\ldots, z_d)$
 	\State \Return $(z_1,\ldots, z_d)\cdot \text{gradOutput}$
 \EndFunction
\end{algorithmic}
\end{algorithm}
\fi

{\bf Complexity.} Computing the DEMD distance in the forward pass is exactly $O(nd)$: linear in the number of distributions and number of bins. This property follows directly from the algorithm, needing only a single pass through all of the data. \cite{BEIN199597} provides this result for greedy algorithms that solve OT programs as in Def.~\ref{def:dmmot}. {\color{blue}  In contrast to methods that derive gradients via entropic regularization schemes, i.e., relaxations of the optimal transport problem \citep{diffpropsinkwass,difftopkOT,quantnorm}, this approach solves the distance computation exactly in linear time.}
This linear time analysis is not only provided by the theory in prior work, but is also explicit in the number of iterations defining our algorithm (see Appendix~\ref{sec:suppdemd} for more details).
For minimizing the DEMD {\color{blue}(computing the updates, i.e., red arrows in Fig.~\ref{fig:min_demd})}, a convergence analysis would follow from the properties of the optimization scheme chosen. Our tool can be dropped in exactly as any other module in modern learning applications (using the observation that gradients are easily computed, i.e., $O(1)$ time to read stored dual variables). %Convergence analysis would follow from the optimization method chosen (SGD, ADAM, etc.).
%}

% It is this construction that we take advantage of and expand upon. 
% \vikas{pose the issue/question first. Then say that we will start from Kline, which we observe can be part of a solution.}

% In this section, we make precise the generalization of the classical Earth Mover's distance. We also state Theorem~\ref{thm:dualgrad}, which establishes that a solution to the generalized linear program's dual equals the gradient of the optimal objective value of the primal linear program.

% \subsection{Multi Marginal Optimal Transport}

% Given a collection of probability distributions $P_1, \ldots, P_m$ over $\cX$, and the cost function $c_m : \cX^m \rightarrow \RR_{\ge 0}$, the multi-marginal optimal transport problem is to minimize $\EE[c_m(X_1,\ldots, X_m)]$ over all couplings of $P_1, \ldots, P_m$, or $X_i \sim P_i$.

% \begin{definition}
% Let $p_1, \ldots, p_m$ be discrete distributions, with each $p_i \in \Delta^n$, where $\Delta^n$ is the $n$-dimensional simplex. Let $c_m : \cX^m \rightarrow \RR_{\ge 0}$. Then the \textbf{discrete multi-marginal optimal transport problem} is
% \begin{align}
%     &\min_{X \in \cX^m}\quad c_m(X) \\
%     &s.t.\quad X_i = p_i \ \forall i
% \end{align}
% \end{definition}

% \begin{proposition}
% Restrict the cost $c_m$ above to be Monge. Then the \textbf{\textit{discretized} multi-marginal optimal transport} is equivalent to the $m$-dimensional earthmover's distance.
% \end{proposition}

% \begin{proposition}
% The d-MMOT or DEMD problem is efficiently solvable.
% \end{proposition}

% \begin{proposition}
% The algorithm results in gradients readily available...
% \end{proposition}


% Before stating the generalized formulation, several points are worthwhile to note. First, the generalized description contains the classical Earth Mover's distance when pairs of distributions are being compared. Second, the generalized program suffers from the curse of dimensionality. If $d$ histograms with $n$ bins are being compared, the decision variable of the program lies in $\RR^{n^{d}}$, and $n^d$ can be very large even if both $d$ and $n$ are modest (say $n=d=10$). 
% However,  this is not an issue in practice: both primal and dual linear programs can be solved using an extremely efficient greedy algorithm, and the decision variable is extremely sparse.

% The classical EMD in (\ref{eq:2demd}) is limited to quantifying dissimilarity between pairs of probability distributions.   But it is useful for applications to be able to break free from this constraint. We describe a framework that matches the classical EMD when pairs of distributions are  being compared, and the framework extends naturally to quantify dissimilarity amongst many probability distributions. To this end, we generalize the Earth Mover's problem stated in the linear program (\ref{eq:2demd}) as follows.  

% {\color{red}:JK motivation, and slow ramp intro}
% Practically, the barycenter is often not even an object of direct interest and instead a means to identifying some direction in which other distributions may be pushed to become closer. Taking advantage of a new algorithm for $d$-dimensional transport problems, we can address a number of the above limitations directly.

% Consider that instead of identifying the barycenter and the sum of pairwise distances, we extend the 2-dimensional problem directly. 

% Let us look closely at the expanded $d$-dimensional earth mover's problem.
% Let $p_1,\ldots, p_d\in\RR^{n}_+$ satisfy $e'p_j=1$ for all $j\in[d]$. 
% Define a cost over the multi-index $(i_1,\ldots, i_d)\in\ZZ^d$ as
% \begin{align*}%\label{eq:cost}
%     c(i_1,\ldots,i_d) = \max\{i_k\}_{k\in[d]} - \min\{i_k\}_{k\in[d]}.
% \end{align*}
% Then, the cost $c\in\RR^{n\times n\times \cdots\times n}$ is a rank $d$-tensor, and when $d=2$, the cost reduces to $c(i,j)=|i-j|$. 
% The generalized Earth Mover's Distance can be written out as 

% The solution to this program effectively identifies a joint distribution $x$ such that the marginals satisfy all $p_j$, and which minimizes the cost. 
% Although the optimal value of the objective function of this program no longer defines a metric when $d>2$, it still possesses several desirable properties.
%, which we use. 
% When used for regularization of machine 
% learning models, a few properties are 
% important to highlight. 

% \subsection{Minimizing dMMOT}
% \ronak{distinction/importance of minimizing the mmot should be made clear above, (shorten/combine sec 3 and 4, shorten intro/related), a central goal of ours is to push dists together, which is hard with other measures.}


% \begin{remark}
% {\bf (1)} The optimal objective value is a continuous function of each probability distribution $p_j$ (i.e., small changes in one distribution cause small changes in the objective value), and 
% {\bf (2)} The optimal objective value vanishes if and only if $p_i=p_j$ for all $i,j\in[n]$.
% \end{remark}
% The first point is critical for numerical stability in applications. The second point is relevant since our goal is to regularize the model in a way that favors equi-distribution of mass across multiple probability densities.

% We can also write the dual linear program.
% \begin{alignat}{2}\label{eq:dualgeneralemd}\begin{split}
% &\underset{z_j\in\RR^n, j\in[d]}{\textrm{maximize}}\qquad\sum_{j} x_j'z_j\\
% &\textrm{subject to}\qquad z_{1}(i_1)+\cdots+z_{d}(i_{d})\leq c(i_1,\ldots,i_{d}),\end{split}
% \end{alignat}
% where the indices in the constraints include all $i_j\in[n]$, $j\in[d]$.

% The following result will make use of the dual linear program.  Before stating the theorem, we introduce some notations. Given distributions $p_1,p_2,\ldots,p_d\in\RR^{n}$ with $e'p_j=1$ for all $j\in[d]$, denote by $\phi(p_1,\ldots,p_d)$, the optimal objective value of the LP in (\ref{eqn:gemd}). Denote by
% \begin{align*}z^*=(z_1^{*}, z^*_2,\cdots,z^*_{d}),
% \end{align*}
% an optimal solution to the dual program~(\ref{eq:dualgeneralemd}).
% \begin{theorem}
% In the established notation, the following two claims hold. First,
% \begin{align*}
% \nabla \phi(p_1,\ldots,p_{d}) = z^*
% \end{align*}
% and second, for any $t\in \RR$,
% \begin{align*}
% \phi(p_1,p_2,\ldots,p_{d}) = \sum_{j}p_j'
% (z_j^* + t\, \eta),
% \end{align*}
% where $\eta\coloneqq (z_1^{*}(n)\,e, z^*_1(n)\,e, \cdots, z^*_{d}(n)\,e)$.
% \label{thm:dualgrad}
% \end{theorem}
% The proof is presented in Section~\ref{sec:proofthm}.

% The second claim, which is rather technical, requires some motivation/explanation for why it was included. We observed empirically that solutions yielded by either autograd or by numerical approximations differed from the solutions yielded by the greedy algorithm that we employ. But the difference was, in all cases, inconsquential: the dual objective functional, evaluated on both vectors agreed, and in all examples tested, the difference equaled a constant factor times the vector $\eta$ that is stated in the theorem.

{\bf A few practical adjustments.}
% In~\cite{kline2019properties}, several properties of the objective functional~$\phi$ are shown. 
% We utilize several of these properties.
% We already observed that $\phi$ is a continuous functional (also stated in Theorem 2.1 of that work). 
% It is also shown there that $\phi$ is homogeneous of degree 1, in the sense that $\alpha\phi(p_1,\ldots,p_d)=\phi(\alpha p_1,\ldots,\alpha p_d)$ for all $\alpha>0$.
It often happens during training that the optimal solutions may, through updates by stepping in the direction of the gradient, acquire entries that are negative. 
This violates an assumption that entries must be nonnegative. However, Thm.~2.2 in~\cite{kline2019properties} shows that the optimal objective value, $\phi$, possesses a type of translation invariance.  We can leverage this result to ensure that, in the event that a point escapes the nonnegative orthant of $\RR^{n}$,
an appropriately constructed constant vector may be added to the current iterate so that it again lies in the nonnegative orthant, \textit{without changing the objective value}. 
Further, with a
% Further, since we wish to compare probability distributions normalized so that $e'p_j=1$ for all $j$, if at some point during training a step modifies $p_j$ so that $e'p_j\not=1$, we can normalize by the positive scalar, $e'p_j>0$. In practice, since each step is a function of a
relatively small learning rate, by the continuity of $\phi$, normalization is a small perturbation of the original point and can be applied as necessary to enforce that updates result in valid distributions. 

\begin{remark}
Another useful property is that if the convex hull of $(p_1,\ldots,p_d)$ is contained in the convex hull of $(\hat{p}_1,\ldots,\hat{p}_{\hat{d}})$, then $\phi(p_1,\ldots,p_d)\leq \phi(\hat{p}_1,\ldots,\hat{p}_{\hat{d}})$, with strict inequality if containment is strict.
A direct consequence of this property is that points within the interior of the convex hull of the data fed to the optimization model have vanishing gradients.
Practically, this leads to \textbf{sparse} gradients w.r.t. the distributions, and nonzero gradients correspond to points on the hull, i.e., distributions which are maximally different from the rest.
Minimization proceeds by iteratively moving mass such that these maximally different distributions are pushed toward each other.
Fig.~\ref{fig:min_demd} shows the convex hull during minimization of the generalized EMD objective via our DEMD algorithm.
\end{remark}

\iffalse % begin removal of histogram wrapfig
\begin{wrapfigure}[15]{R}{0.45\textwidth}
    \vspace{-60pt}
    \begin{minipage}{0.45\textwidth}\vspace{2.5\baselineskip}
    \begin{algorithm}[H]
        \small
        \caption{Differentiable Histograms}\label{alg:diffhist}
        \begin{algorithmic}
        \Function {Init}{$n$} %{\em \hspace*{\fill} // discretization level}
            \State $r := 1/n$;\  %{\em \hspace*{\fill} // bin size};
             $bounds := [0, r, 2r, \ldots, 1]$ %{\em \hspace*{\fill} // bin boundaries}
        \EndFunction
        \Function {Forward}{$acts$}
            \State $counts = []$; $\mathit{cdfs} = \sigma(acts)$ %{\em \hspace*{\fill} // compute CDFs via Sigmoid}
            
            \For{$b \in bounds$}
                \State $cnt = 0$; $dist = \lvert \mathit{cdfs} - b \rvert$ %{\em \hspace*{\fill} // dist. to boundary}
                \For{$i \in [1,\ldots,n]$}
                    \State $cnt = cnt + \operatorname{ReLU}(r - dist[i])$ %{\em \hspace*{\fill} // soft bucket count}
                \EndFor
                \State $counts[b] = cnt$
            \EndFor
            \State $out = counts/sum(counts)$
            \State \Return $out$
        \EndFunction
        \end{algorithmic}
    \end{algorithm}
    \end{minipage}
    %\vspace{-30pt}
\end{wrapfigure} 
% % DO NOT DELETE THE following BLANK LINE 
% % IT IS REQUIRED TO SOLVE A 2-column issue
% % connected to wrapfigure

% \phantom{.} % DO NOT DELETE -- it is required to solve wrapfigure 2-column bug


\fi % end removal of histogram wrapfig

% \subsection{Solving the generalized EMD program and its dual}
% Linear program solvers are now a  mature technology,  with polynomial-time guarantees for solution accuracy.
% Despite this, they are unsuitable as the engine that solves the programs of our intended use-case, namely regularizing a network model. 
% Our intent is to have the ability to embed a process that solves an instance of (\ref{eqn:gemd}) and its dual repeatedly during network training.  One reason standard methods are too slow for this is that the programs themselves can be quite large: recall that the cost functional is a dense, rank-$d$ tensor. We note 
% however that there is recent progress towards differentiable LP modules  \cite{meng2020differentiable,NEURIPS2020_6bb56208}. 

% With a single pass over all input histograms, both the primal and dual decision variables are computed in a sparse manner. The absence of division or scaling also provides numerical stability.
% This simply means the number of bins in histogram $p_i$ need not match the number of bins in $p_j$.



% \begin{algorithm}[ht!]
% \caption{EMD Primal/Dual Algorithm\jeff{do not need to show greedy algorithm here}}\label{alg:primaldual}
% \begin{algorithmic}
% \State {\bf input} $p_j\in\RR^{n}_+$ with $e'p_j=1$ , $(\forall j\in[d])$
% % \State {\bf initialize}
% \State {\em //iteration index, active indices, Eq.~\eqref{eq:gemd} variable, Eq.~\eqref{eq:dualgeneralemd} variable}
% \State  $k \coloneqq 0\in\ZZ$, $I \coloneqq 0\in\ZZ^{d}$, $x \coloneqq 0\in\RR^{n^{d}}$, $z_j\coloneqq 0\in\RR^{n}$, $(\forall j\in [d])$
% % \State  $I \coloneqq 0\in\ZZ^{d}${\em \hspace*{\fill} // track the active indices}
% % \State  $x \coloneqq 0\in\RR^{n^{d}}${\em \hspace*{\fill}  // variable of (\ref{eq:gemd})}
% % \State  $z_j\coloneqq 0\in\RR^{n}$, $(\forall j\in [d])$ {\em \hspace*{\fill}  // variable of (\ref{eq:dualgeneralemd})}
% \While{$I(j) \leq n$, $(\forall j\in[d])$}
%     \State $s_k\coloneqq \min_{j\in[d]} p_j(I(j))$ {\em\hspace*{\fill}  // the mass to move}
%     \State $x(I)\gets s_k$  {\em \hspace*{\fill} // update the EMD solution}
%     \State $p_j(I(j))\gets p_j(I(j))-s_k$, $(\forall j\in d)${\em\hspace*{\fill} // shrink the data}
%     \State $j^*\gets\arg\min_{j\in[d]} p_j(I(j))$
%     \State  $I(j^*)\gets I(j^*)+1$
%     \State $k\gets k+1$
%     \State $t_k\gets c(I)${\em\hspace*{\fill} // cost of this step}
%     \If{ $I(j^*)\leq n$}
%         \State  $z_{j^*}(I(j^*))\gets t_k-t_{k-1} + z_{j^*}(I(j^*)-1)$ {\em\hspace*{\fill} // update the dual solution}
%     \EndIf
% \EndWhile
% \State \Return $x$, $(z_1,\ldots,z_d)$, and the objective value $\sum_k s_k t_k$.
% \end{algorithmic}
% \end{algorithm}

% Algorithm~\ref{alg:primaldual} describes the greedy algorithm that solves both primal and dual generalized Earth mover's programs. The algorithm accepts $d$ distributions (i.e., histograms) $p_1,\ldots,p_d\in\RR^{n}_+$ with $e'p_j=1$ for all $j\in[d]$. 
% Although the algorithm states that all histograms have the same number of bins, the algorithm can be 
% easily adapted to accept as inputs $p_i\in\RR^{n_i}_+$ with $n_i\not=n_j$. 

% \subsection{Minimizing d-MMOT within a loop}
% With a computable generalized EMD objective in hand,
% we now move towards
% %minimizing the d-MMOT problem
% %in modern neural network pipelines.
% %The objective is a concurrent
% the minimization of the appropriate learning loss $L(f(x),y;\theta)$ alongside distances among different distributions $g_i\in G, i \in [d]$, $\text{d-MMOT}(f(x),g;\theta)$, i.e., 
% $\min_{\theta} L(f(x),y;\theta) + \text{d-MMOT}(f(x),g;\theta) $.
% However, note 
% that outputs $f(x)$ and intermediate activations are \textit{continuous} values (layer shape, batch size).
% So, we must transform activations into normalized histograms (i.e., discrete distributions).
% Only once histograms have been generated for different groups of interest can we apply our algorithm and minimize the distances between groups.}

% want to reg. solution found by NNs to a common distribution over all groups.
% common distribution is found with DEMD
% requires computing it
% 
%
% {\bf Precomputed dual variables.} The first issue is trivial to implement within standard neural network frameworks. The ``forward" pass of the EMD module can be defined using Alg.~\ref{alg:primaldual}, with one minor modification.  Prior to returning, the dual solution variables are stored for use during the backward pass. As backpropagation proceeds, when the EMD module encounters an incoming gradient, it is simply multiplied by the stored dual variables (see Algorithm~\ref{alg:demdFunc}).
% \begin{algorithm}
% \caption{Pseudocode of the generalized EMD objective module.}\label{alg:demdFunc}
% % \begin{algorithmic}
% \SetKwFunction{FF}{forward}
% \SetKwProg{FF}{forward}{:}{}
% \FF{($p_1, \ldots, p_j$)}{
%         Get $\sum s_k t_k$ and $(z_1,\ldots,z_d)$ from Algorithm \ref{alg:primaldual}. \\
%         \text{save} $(z_1,\ldots, z_d)$ \\
%         \KwRet $\sum_k s_k t_k$
%   }
% \SetKwFunction{FB}{backward}
% \SetKwProg{FB}{backward}{:}{}
% \FB{(gradOutput)}{
%     \text{load} $(z_1,\ldots, z_d)$ \\
%         \KwRet $(z_1,\ldots, z_d)*gradOutput$
%   }
% \end{algorithm}
% \begin{wrapfigure}[18]{r}{0pt}%{0.45\textwidth}
%     \begin{minipage}{0.45\textwidth}\vspace{2.5\baselineskip}
%     \begin{algorithm}[H]
%         \small
%         \caption{Differentiable Histograms}\label{alg:diffhist}
%         \begin{algorithmic}
%         \Function {Init}{$n$} %{\em \hspace*{\fill} // discretization level}
%             \State $r := 1/n$ %{\em \hspace*{\fill} // bin size}
%             \State $bounds := [0, r, 2r, \ldots, 1]$ %{\em \hspace*{\fill} // bin boundaries}
%         \EndFunction
%         \Function {Forward}{$acts$}
%             \State $cdfs = \sigma(acts)$ %{\em \hspace*{\fill} // compute CDFs via Sigmoid}
%             \State $counts = []$
%             \For{$b \in bounds$}
%                 \State $dist = \lvert cdfs - b \rvert$ %{\em \hspace*{\fill} // dist. to boundary}
%                 \State $cnt = 0$
%                 \For{$i \in [1,\ldots,n]$}
%                     \State $cnt = cnt + \operatorname{ReLU}(r - dist[i])$ %{\em \hspace*{\fill} // soft bucket count}
%                 \EndFor
%                 \State $counts[b] = cnt$
%             \EndFor
%             \State $out = counts/sum(counts)$
%             \State \Return $out$
%         \EndFunction
%         \end{algorithmic}
%     \end{algorithm}
%     \end{minipage}
%     %\vspace{-30pt}
% \end{wrapfigure} 
% % DO NOT DELETE THE following BLANK LINE 
% % IT IS REQUIRED TO SOLVE A 2-column issue
% % connected to wrapfigure

% \phantom{.} % DO NOT DELETE -- it is required to solve wrapfigure 2-column bug

{\bf Need for Histogramming.} 
{\color{blue} Note that outputs $f(x)$ and intermediate activations are \textit{continuous} values (layer shape by batch size).}
So, we must transform activations into normalized histograms (i.e., discrete distributions).
While libraries such as PyTorch and Tensorflow typically provide histogram functions, they are not differentiable. The discontinuous operation of bin assignment does not allow for an end-to-end pipeline where we can push upstream parameters in the direction of minimizing the EMD objective over histograms.
To address this, we 
use a simple relaxed/differentiable histogramming operation. 
First, all outputs are mapped to $[0,1]$ using a Sigmoid activation. Then, for each bin location, the ``count'' in that bin is determined by a ReLU function applied to the difference between the gap between the activation and the distance to the bin boundary. In other words, if an activation falls in a bin, the count of that bin increases based on the distance to the bin boundary, otherwise the count remains the same.
The full procedure is detailed in Appendix~\ref{sec:supphist}.
The ReLU activations defining bin boundaries allow for gradients to move samples towards neighboring bins as needed.

% \noindent\begin{minipage}{.49\textwidth}
% \begin{algorithm}
% % \caption{Differentiable Histograms}\label{alg:diffhist}
% \begin{algorithmic}[1] \footnotesize
% \Function {Init}{$n$} {\em \hspace*{\fill} // #bins, (discretization)}
%     \State $r := 1/n$ {\em \hspace*{\fill} // bin size}
%     \State $locs := arange(0, 1, r)$ {\em \hspace*{\fill} // bin boundaries}
% \EndFunction
% \Function {Forward}{$acts$}
%     \State $cdfs = \sigma(acts)$ {\em \hspace*{\fill} // compute CDFs}
%     \State $counts = []$
%     \For{loc in locs}
%         \State $dist = \lvert cdfs - loc \rvert$ {\em \hspace*{\fill} // dist. to boundary}
%         \State {\em // soft bucket count}
%         \State $ct = \underset{i\in[nbins]}{\sum} \operatorname{ReLU}(r - dist[i])$ {\em \hspace*{\fill} // soft bucket count}
%         \State $counts.append(ct)$
%     \EndFor
%     \State $out = stack(counts)$
%     \State $out = out/sum(out)$
%     \State \Return $out$
% \EndFunction
% \end{algorithmic}
% \end{algorithm}
% \captionof{algorithm}{test}
% \end{minipage}%
% \begin{minipage}[t]{.49\textwidth}
%   \centering
%   \includegraphics[trim={8cm 0 8cm 0},clip,width=\textwidth]{figs/net_diff.png}
%   \captionof{figure}{This is another figure.} \label{myfig2}
% \end{minipage}



% While gradients are now readily available, typical machine learning pipelines do not have distributions or histograms predefined at outputs which can be fed directly into our EM Loss. Applying existing binning procedures over the batch to estimate histograms will break the ability to autodifferentiate: soft thresholds are necessary at bin boundaries such that samples within a bin may move smoothly as needed. Algorithm \ref{alg:diffhist} provides a differentiable histogram implementation. Using a rectified linear relaxation allows for samples to have a continuous gradient towards neighboring bins.


% \subsection{Minimizing over Multiple Dimensions}
% {\color{red} @Ronak - shall we delete this section?}
%     -extending past simple computation
%     -minimizing DEMD: forcing distributions to become similar wrt to distance
%     -if possible, would lead to another, fast, way to push distributions together/towards "barycenter"

%     minimizing GEM
%     at first does not seem different from existing wass bary approaches
%     requires differentiating through distance computation, updating many times
%     -however particular algorithm of above leads to direct identification of gradients
%     -no backwards derivations/complex derivatives necessary

% \subsection{The dual linear program and back propigation}
% {\color{red}@Ronak - do you want to place backprop comments here or below?}
% The above allows for a direct plug in to existing frameworks. By manually defining the backward pass using the dual variables, the forward pass can be written explicitly and the dual variables stored for backpropagation. 

% \ronak{Will add algorithmic block detailing forward/backward, example in loss/reg formulation}


% Procedure 1:
% \begin{enumerate}
%     \item Given a trained model $\theta$,
%     \item compute the d-EMD distance for all groups
%     \item identify the groups with maximal ``unfairness" via the hull in \eqref{eq:dEMD}.
%     \item Take a coordinate step in reducing the gap.
%     \item Repeat.
% \end{enumerate}


%% This declares a command \Comment
%% The argument will be surrounded by /* ... */
% \SetKwComment{Comment}{/* }{ */}
% \begin{algorithm}
% \caption{Algorithm Pseudocode}\label{alg:main}
% \KwData{$X, Y, G$, algorithm $f$ that learns $\theta$}
% \KwResult{Fair model $\theta$}

% \KwFor{stopping cond} {
%     $\theta_{next} = f(X,Y, G_{prev}, \theta_{prev})$\;
%     Compute EMD for all $G$.\;
%     $G_{next}$ = Select maximally unfair groups via \eqref{eq:floormin}\;
%     $\theta_{prev} = \theta$\;
%     $G_{prev} = G_{next}$\;
% }
% \end{algorithm}
% \input{_practice}
\section{L-FOCI in Generic ML Settings}
\begin{figure}
    \centering
    \includegraphics[width=0.98\columnwidth,trim={0, 0, 0, 0},clip]{5_unlearn/figs/Speed_Hist.png}
    \caption[Speedups from L-CODEC randomization]{\label{fig:speed_hist} L-CODEC vs CODEC run time comparison for identifying sufficient subsets for each CelebA attribute separately (pairs of columns, details in supplement).}
\end{figure}
We begin with understanding the value of L-CODEC and L-FOCI for Markov Blanket Identification and progress to applications
% Once their use has been properly justified in simpler, more direct applications, we demonstrate the value directly 
in typical unlearning tasks involving large neural networks previously infeasible with existing scrubbing tools.
\todo{See appendix for additional details.} 
%Our code is at \url{https://github.com/xxx-xxxx/}.

% \subsection{L-FOCI in Generic ML Settings}
% \begin{figure}
%     \centering
%     %\includegraphics[width=0.25\textwidth]{figs/Pearson_noise_0.png}
%     \includegraphics[width=0.3\textwidth]{figs/Spearman_noise_0.png}
%     %\includegraphics[width=0.45\textwidth]{figs/speed_hist.png}
%     \includegraphics[width=0.3\textwidth]{figs/CODEC_noise_0.png}
%     \includegraphics[width=0.3\textwidth]{figs/CODEC_noise_0.01.png}
%     \caption{Correlation matrices over the attributes in CelebA using (left) Spearman rank correlation, (middle) FOCI with CODEC, and (right) L-FOCI with  L-CODEC.}
%     \label{fig:corrmats}
% \end{figure}
\paragraph{L-CODEC Evaluation.}
To assess speedup gained in the discrete setting when running L-CODEC,
% We first demonstrate the additional speedup gained in the discrete setting when running L-CODEC.
we construct the Markov Blanket for specific attributes provided as side information with the CelebA dataset. Figure~\ref{fig:speed_hist} shows the wall-clock times for Markov Blanket Selection via FOCI and L-FOCI for each attribute.
%{\color{red}In experiments, we use L-CODEC where discrete values or tiebreaks are common, and traditional CODEC when values are continuous or unique.}

\paragraph{Markov Blanket Identification.} We replicate the experimental setup in Section 5.3 of \cite{bullseye}, where a high dimensional distribution over a ground truth graph is generated, and feature mappings are used to reduce the dimension and map to a latent space. Table~\ref{tab:bullseye} summarizes subset identification efficacy and runtime. Replacing conditional mutual information (CMI) with L-CODEC, we see a clear improvement in both runtime and Markov Blanket identification over the raw data, and comparable results in the latent feature space. Using L-FOCI directly in the feature space, we identify an additional spurious feature not part of the Markov Blanket, but runtime is significantly faster.
\begin{table}
    \centering
    \begin{tabular}[b]{l|ccr|ccr}
        \hline\hline
        & \multicolumn{3}{c|}{Raw Data} & \multicolumn{3}{c}{Feature Maps} \\
        Method & TPR & FPR & Time (s) & TPR & FPR & Time (s) \\
        \hline
        \cite{bullseye} & 0.75 & 0.50 & 5124.22 & \textbf{0.875} & \textbf{0.00} & 516.19 \\
        L-CODEC + CIT & \textbf{1.00} & \textbf{0.50} & \textbf{402.10} & 0.75 & \textbf{0.00} & 117.29 \\
        L-CODEC + L-FOCI & \multicolumn{3}{c|}{N/A} & 0.833 & 0.50 & \textbf{0.464} \\
        \hline\hline
        % & \multicolumn{3}{c}{Feature Maps} \\
        % & TPR & FPR & Time (s) \\
        % \hline
        % \cite{bullseye}  & \textbf{0.875} & \textbf{0.00} & 516.19 \\
        % L-CODEC + CIT & 0.75 & \textbf{0.00} & 117.29  \\
        % L-CODEC + L-FOCI & 0.833 & 0.50 & \textbf{0.464} \\
        % \hline\hline
    \end{tabular}
    \caption[Markov blanket identification]{\label{tab:bullseye} 3D-Bullseye Markov Blanket identification. CIT represents the model in \cite{bullseye}. Both L-CODEC and L-FOCI run much faster than recent Markov Blanket identification schemes. L-FOCI is not applicable to the multi-dimensional raw data setting.}
\end{table}
% \begin{table}
% \centering
% \begin{tabular}{l|cccccc}
%     \hline\hline
%     & \multicolumn{3}{c}{Raw Data} & \multicolumn{3}{c}{Feature Maps} \\
%     & TPR & FPR & Time (s) & TPR & FPR & Time (s) \\
%     \hline
%     CMI + CIT \cite{bullseye} & 0.75 & 0.50 & 5124.22  & \textbf{0.875} & \textbf{0.00} & 516.19 \\
%     L-CODEC + CIT & \textbf{1.00} & \textbf{0.50} & \textbf{402.10} & 0.75 & \textbf{0.00} & 117.29  \\
%     L-CODEC + L-FOCI & -- & -- & -- & 0.833 & 0.50 & \textbf{0.464} \\
%     \hline\hline
% \end{tabular}
% \begin{tabular}{l|ccc}
%     \hline\hline
%     & \multicolumn{3}{c}{Raw Data} \\
%     & TPR & FPR & Time (s) \\
%     \hline
%     CMI + CIT \cite{bullseye} & 0.75 & 0.50 & 5124.22 \\
%     L-CODEC + CIT & \textbf{1.00} & \textbf{0.50} & \textbf{402.10} \\
%     L-CODEC + L-FOCI & -- & -- & -- \\
%     \hline\hline
% \end{tabular}
% \begin{tabular}{l|ccc}
%     \hline\hline
%     & \multicolumn{3}{c}{Feature Maps} \\
%     & TPR & FPR & Time (s) \\
%     \hline
%     CMI + CIT \cite{bullseye}  & \textbf{0.875} & \textbf{0.00} & 516.19 \\
%     L-CODEC + CIT & 0.75 & \textbf{0.00} & 117.29  \\
%     L-CODEC + L-FOCI & 0.833 & 0.50 & \textbf{0.464} \\
%     \hline\hline
% \end{tabular}
% \caption{\label{tab:bullseye} 3D-Bullseye Markov Blanket identification. CIT represents the conditional independence testing framework used in \cite{bullseye}. Both L-CODEC and L-FOCI run significantly faster then recent Markov Blanket identification schemes.}
% \end{table}

% Setup 2 (Complex Dependencies): previous is somewhat ``proprotional". Describe DAG from random base operations (Ronak's synthetic. Same eval as above.



% In this set of experiments, we take advantage of the selective power of conditional independence scores to identify a set of spurious features. We think of spurious features as ones that can maliciously influence the decision making process of the models being learned by virtue of their non-trivial correlation with the target attribute. In the experiments presented in this paper, we consider the CelebA dataset and consider the task of binary classification for certain attributes which we decide based on our observations of spurious features. 
% We consider the prediction of attributes: ``Wearing Lipstick", ``Young", ``No Beard" and ``Smiling" in four different experiments. 
% The key step is to identify the spurious features for each of these attributes separately.
% For that, we use the L-CODEC algorithm described above with varying amounts of noise. 
% Once the spurious attributes are identified, we use them in a Gradient Reversal framework for regularizing against the spurious attributes, while predicting the specific target attribute.
% The prediction model being considered in these experiments is the MobileNet-V2. We use this architecture for a 2-class prediction
% problem and train it from scratch. The experimental results are presented in the table and discussed below.
% When using the full set of attributes, we significantly reduce performance, even with a number of different parameter settings.
% Unregularized, we can clearly see a biased model with respect to the attributes of interest.



%\begin{outline}[enumerate]
%\1 describe synthetic graph setup
%    \2 describe results
%    \2 immediate identification of markov blanket
%\1 describe bullseye setup (2d/3d)
%    \2 replacing statistic
%    \2 replacing entire procedure
%    \2 describe results
%    \2 significantly faster with similar results
%\1 What can we do with it? CelebA
%    \2 identify markov blanket over specific attribute vs others
%    \2 add invariance via gradient reversal
%    \2 compare performance on full set and conditioned set when small set and full set of features is conditioned
%        \3 full set is too bad, drastic drop in acc
%        \3 select only spurious features
%\end{outline}

% \input{_explain.tex}

\begin{figure}
    \centering
    \includegraphics[width=0.155\columnwidth,trim={0.25in 0 10.25in 0},clip]{5_unlearn/figs/No_Beard_All.png}
    \includegraphics[width=0.25\columnwidth,trim={6.5in 0 0 0},clip]{5_unlearn/figs/No_Beard_All.png}
    \includegraphics[width=0.25\columnwidth,trim={6.5in 0 0 0},clip]{5_unlearn/figs/No_Beard_Random.png}
    \includegraphics[width=0.25\columnwidth,trim={6.5in 0 0 0},clip]{5_unlearn/figs/No_Beard_FOCI.png}
    \caption[Spurious feature regularization with L-FOCI]{\label{fig:spur} Validation accuracies after training to predict ``No Beard" in the CelebA dataset. (L to R) regularization for all features, for a random subset, and via FOCI. Green indicates accuracy on the data with that feature, red, without.}
\end{figure}

\paragraph{Spurious Feature Regularization.}
This Markov Blanket $(MB)$ identification scheme can be used to address spurious feature effects on traditional NN models.
%In modern machine learning applications, strict classification or regression on outcomes of interest is no longer sufficient to deploy models in the real world. 
%Particularly, a large number of existing machine learning problems such as overfitting to spurious features and class imbalance can drastically affect model performance on biased training data. 
% Removing the effect of spurious features can be achieved in many ways, but it requires knowledge of which features should be considered.
% Here, we look at applications in which there may be a large number of outside factors, or side information, that may be heavily dependent, or spurious, to the outcome variable. 
%Using typical tools in the ML toolbox,
% A natural method of attempting to incorporate side information would be to add appropriate loss terms to the global objective, where the loss terms may be pushing the learning algorithm towards \textit{regularized} models that are further away from the training set optimum, but closer to the solution of the true question.
A straightforward approach would be to directly add a loss term for each potentially important feature over which we would like to regularize, 
$\cL(\theta) + \sum_{S\in \cS} R_S(\theta)$.
However, with a large number of outside factors $S$, this can adversely effect training.
% ,to the point where for any reasonable regularization parameter settings no model is able to be found with high performance on the outcome of interest. 
We instead use L-FOCI to identify the set of minimal factors that, when conditioned, make the rest conditionally independent. Then it is only necessary to include regularizers over $S \in MB(Y)$.
%Here, if we consider a graph over the external factors and our outcome of interest, it is immediately clear that if we wish to regularize the effect of all factors, it is sufficient to only regularize the effect of the Markov Blanket. 
% \begin{align}
% \cL(\theta) + \sum_{S\in MB(Y)} R_S(\theta)
% \end{align}

We evaluate a simple attribute image classification setting using the CelebA dataset. We run L-FOCI over the attributes as in our L-CODEC evaluation, and regularize using a Gradient Reversal Layer for a simple accuracy term over those attributes.
% We compare models with varying regularization over all attributes, randomly chosen attributes, and FOCI-selected attributes.
Results in Figure \ref{fig:spur} clearly show that selection with FOCI provides the best result, maintaining high overall accuracy but also preserving high accuracy on sets of samples with/without correlated attributes.

\section{L-FOCI for Machine Unlearning}

\subsection{Comparing to Full Hessian Computation}
For simple regressors, we can compute the full Hessian and compare results generated by a traditional unlearning update, our L-FOCI update, and a random selection update. To reduce variance and show the best possible random selection, we run our L-FOCI and randomly choose a set of the same size for each random selection. Figure~\ref{fig:mnistcifar} (left) shows validation and residual accuracies for 1000 random removals from MNIST (averaged over 10 runs). 

\begin{figure}
    \centering
    %\includegraphics[width=0.3\textwidth]{figs/scrub/MNIST_Valid_Acc.png}
    \includegraphics[width=0.24\textwidth]{5_unlearn/figs/scrub/MNIST_Resid_Acc.png}
     \includegraphics[width=0.24\textwidth]{5_unlearn/figs/scrub/MNIST_GradNorm_Logistic.png}
    % \label{fig:mnist}
    % \includegraphics[width=0.33\textwidth]{5_unlearn/figs/scrub/cifar_val.png}
    \includegraphics[width=0.24\textwidth]{5_unlearn/figs/scrub/cifar_resid.png}
    \includegraphics[width=0.24\textwidth]{5_unlearn/figs/scrub/CIFAR_gradnom.png}
    \caption[Residual accuracies and gradient norms]{(Left) Residual Accuracies \& Sample Gradient Norm of removal for an MNIST Logistic Regressor. Averaged over 10 runs. (Right) Residual accuracies and sample gradient norms for various CIFAR-10 models.}
    \label{fig:mnistcifar}
\end{figure}


\paragraph{Are we selecting reasonable subsets?} A natural question is whether the subset selection via L-FOCI is any better than random, given that we are effectively taking a smaller global step. We answer this in the affirmative with a simple comparison with a random selection of size equal to the set selected by L-FOCI. Figure~\ref{fig:mnistcifar} (left) shows that the sample gradient norm for selections made by L-FOCI are larger than those of a random selection: the subset of the model scrubbed of this specific sample has a larger impact on its final loss, and thus the gradient norm post-removal is large.


% \subsubsection{$\epsilon$-Certified Removal} 
% While we focus on the population minimizer setting from \cite{sekhari2021remember}, we also demonstrate the efficacy of LFOCI as a drop-in tool for another common unlearning setting described in \cite{guo2019certified}: hen a model is trained in a particular manner, a direct Newton ascent step is sufficent to guarantee $\epsilon$-removal. 

% Large drops in validation and residual accuracy indicate that a set of parameters integral to the model's predictions overall have been updated significantly. This suggests that the sample scrubbed may be a ``support vector" for the model. When these samples are considered for scrubbing, the model may have to be retrained. 

\paragraph{Does the formulation scale?} We scrub random samples from various CIFAR-10 models, and evaluate  performance for the same set of hyperparameters. When the models are larger than logistic regression, it is infeasible to estimate the full Hessians, so 
we {\em must} use our L-FOCI selection update. Figure~\ref{fig:mnistcifar} (right) shows removal performance over many typical models with varying sizes. Models that have higher base accuracies tend to support more removals before performance drops. This matches results for differentially private models: models that generalize well may not have overfit and thus may already be private, allowing ``fast'' forgetting.

% \begin{figure}
%     \centering
%     % \includegraphics[width=0.33\textwidth]{5_unlearn/5_unlearn/figs/scrub/cifar_val.png}
%     \includegraphics[width=0.49\columnwidth]{5_unlearn/figs/scrub/cifar_resid.pdf}
%     \includegraphics[width=0.49\columnwidth]{5_unlearn/figs/scrub/CIFAR_gradnom.pdf}
%     \vspace{-8pt}
%     \caption{Residual accuracies and sample gradient norms for various CIFAR-10 models.}
%     \label{fig:cifar}
% \end{figure}

\paragraph{Tradeoff vs Retraining.}
While our focus is the setting in which retraining is not feasible, where we can retrain we compare validation accuracies as a function of number of removals. Using a subset of MNIST, we train to convergence and iteratively remove samples using our construction, retraining fully at each step for comparison. With 1000 training samples from each class 
and reasonable settings of privacy parameters ($\epsilon=0.1,\delta=0.01$),
we support a large percentage of removals until validation accuracy drops more than a few percent, see Figure~\ref{fig:retrain}.
\begin{figure}
    \centering
%\includegraphics[width=0.31\columnwidth]{5_unlearn/figs/retrain/Retrain_Residual_Accs.png}
\includegraphics[width=0.49\columnwidth]{5_unlearn/figs/retrain/Retrain_Validation_Accs.png}
\includegraphics[width=0.49\columnwidth]{5_unlearn/figs/retrain/Retrain_Gradient_Norms.png}
    \caption[MNIST retraining comparison]{MNIST Retraining comparison averaged over 8 runs. Validation accuracies and residual gradient norms.}
    \label{fig:retrain}
\end{figure}

\subsection{Removal in NLP models}
We now scrub samples from transformer based models using  LEDGAR \citep{tuggener2020ledgar}, a multilabel corpus of legal provisions in contracts. We use the prototypical subset which contains $110156$ provisions pertaining to $13$ most commonly used labels based on frequency. Our model is a fine-tuned DistilBERT \citep{sanh2019distilbert} and uses the $[CLS]$ token as an input to the classification head. 
Table \ref{tab:nlp} shows results of scrubbing the provisions from two different classes; \textit{Governing Laws} and \textit{Terminations} which have the highest/lowest support in the test set. As expected with increasing $\epsilon$, i.e., lower privacy guarantees, we can support more number of removals based on the Micro F1 score of the overall model. 
% {\color{red}The Micro F1 scores for the overall model, start to fall off after that of the class being removed and the change in overall scores is more gradual than that of the removal class. }
The Micro F1 scores, for the removed class fall off rapidly, while the change in overall scores is more gradual. 

% \begin{table}
%     \centering
%     \resizebox{!}{60pt}{%
%     \begin{tabular}[b]{l|ccc}
%         \hline\hline
%         & \multicolumn{2}{c}{\# Supported Removals}  \\
%         $\epsilon$ & Governing Laws & Terminations \\
%         \hline
%         0.1 & $>$ 100 & $>$ 100 \\
%         0.01 & $>$ 100 & $>$ 100 \\
%         0.001 & 18 & 21 \\
%         0.0005 & 6 & 7 \\
%         \hline\hline
%     \end{tabular}%
%     }
%     \vspace{-8pt}
%     \caption{\label{tab:nlp} Scrubbing transformer model for provision classification.}
% \end{table}

% \subsubsection{Appendix: Direct Application in existing CR Setting}
% using FOCI in a naive setting does not harm final result. (Equivalent to full when doing one vs. all, better than random).

\subsection{Removal from Pretrained Models}
The above settings show settings where a sample from one specific source may be removed. A more direct application of unlearning 
is completely removing samples from a specific class; a compelling use case is face recognition.

We utilize the VGGFace dataset and model, pretrained from the original work in \cite{huang2008labeled,Parkhi15}. The model uses a total of approximately 1 million images to predict the identity of 2622 celebrities in the dataset. Using a reconstructed subset of 100 images from each person, we first fine-tune the model on this subset for 5 epochs, and use the resultant models as estimates of the Hessian. 
In this setting, the VGGFace model is very large, including a linear layer of size $25088 \times 4096$. Selecting even a few slices from this layer results in a Hessian matrix unable to fit in typical memory. For this reason, we run a ``cheap'' version of L-FOCI: we select only one slice that results in the largest conditional dependence on the output loss.
%
% \begin{figure}
%     \centering
%     \includegraphics[width=0.5\columnwidth]{5_unlearn/figs/scrub/VGG_Scrub_1.pdf}
%     \vspace{-10pt}
%     \caption{Accuracies on scrubbed class and Residual Sample Sets (every 10 removals) for $\epsilon = 1e^{-5}$. The accuracy drop for the residual set is gradual up to a certain number of removals.}
%     \label{fig:vgg}
% \end{figure}

\stepcounter{table}
\begin{figure}
\begin{subfigure}{0.45\columnwidth}\centering
    \includegraphics[width=0.98\columnwidth]{5_unlearn/figs/scrub/VGG_Scrub_1.png}%
    \phantomsubcaption
    \label{fig:vgg}
\end{subfigure}
\begin{subfigure}{0.45\columnwidth}
    \centering
    \begin{tabular}[b]{l|ccc}
        \hline\hline
        %& \multicolumn{2}{c}{}  \\
        & \multicolumn{2}{c}{\# Supported Removals}  \\
        $\epsilon$ & Governing Laws & Terminations \\
        \hline
        0.1 & $>$ 100 & $>$ 100 \\
        0.01 & $>$ 100 & $>$ 100 \\
        0.001 & 18 & 21 \\
        0.0005 & 6 & 7 \\
        \hline\hline
    \end{tabular}%
    \phantomsubcaption
    \label{tab:nlp}
\end{subfigure}
%\captionsetup{labelformat=andtable}

\caption[Removal performance in pretrained models]{(Left) Scrubbed and Residual Accuracies (every 10 removals) for $\epsilon = 1e^{-5}$. The accuracy drop for the residual set is gradual up to a certain number of removals. (Right) Scrubbing transformer model for provision classification.}
\label{fig:combine}
\end{figure}

Figure~\ref{fig:vgg} show results for scrubbing consecutive images from one individual in the dataset for a strong privacy guarantee of $\epsilon=10^{-5}$. As the number of samples scrubbed increases, the performance on that class drops faster than on the residual set, exactly as desired.

\subsection{Removal from a Person re-identification model}
As a natural extension to our experiments on face recognition, we evaluate unlearning of deep neural networks trained for person re-identification. Here, the task is to associate the images pertaining to a particular individual but collected in diverse camera settings, both belonging to the same camera or from multiple cameras. In our experiments, we use the Market-1501 dataset \citep{zheng2015scalable} and a Resnet50 architecture which was trained for the task. We unlearn samples belonging to a particular person, one at a time, and check the performance of the model. Experimental results are in agreement with results reported for the transformer model as well as the VGGFace model. With very small values of $\epsilon$ i.e. $0.0005$ the number of supported removals is limited to less than $10$ depending on the person id being removed. However, with a larger value of $\epsilon$, e.g., $0.1$, all potential samples can be removed without a noticeable degradation in model performance in terms of mAP scores. In Figure~\ref{fig:reid}, we clearly see that after scrubbing a model for a particular person, its predictions for that particular individual become meaningless whereas the predictions on other classes are still possible with confidence, as desired. Additional experiments with different datasets, model architectures and other ablations for deep unlearning for person re-identification models are presented in the \todo{appendix}.

\begin{figure}[]
    \centering
    %\includegraphics[width=0.3\textwidth]{5_unlearn/figs/scrub/MNIST_Valid_Acc.png}
    \includegraphics[width=0.49\columnwidth]{5_unlearn/figs/scrub/0006_c3s3_075694_00.jpg}
     \includegraphics[width=0.49\columnwidth]{5_unlearn/figs/scrub/0101_c5s1_021926_00.jpg}
    \caption[Visualizing unlearning results in person REID models]{\label{fig:reid}Activation maps from a model scrubbed for the person on the left (right set is not scrubbed). For each triplet, from (L to R) are the original image, the activation map and its image overlay. Note the effect of scrubbing: activations change significantly for the scrubbed sample (compare column 2 to 3) whereas remain stable for the non-scrubbed sample (compare column 5 to 6).}
\end{figure}

% \subsubsection{Scrubbing Reveals Sample Influence.}
% Observe in some of above results there are some significantly large changes in sample gradient norms, corresponding to large drops in residual and validation accuracy. These may indicate highly influential samples, in the sense that they may represent ``support vectors" for the model. For dataset X, we show a few of these identified in Figure~\ref{fig:samps}. Qualitatively these appear to be reasonably unique samples for their particular class. While this is observation appears reasonable, the direct connection to sample influence in the traditional sense remains open for future work.


% \begin{figure}
%     \centering
%     \includegraphics[width=0.22\textwidth]{5_unlearn/figs/scrub/custom_full_acc.png}
%     \includegraphics[width=0.22\textwidth]{5_unlearn/figs/scrub/custom_full_grad.png}
%     \includegraphics[width=0.22\textwidth]{5_unlearn/figs/scrub/custom_foci_acc.png}
%     \includegraphics[width=0.22\textwidth]{5_unlearn/figs/scrub/custom_foci_grad.png}
%     \caption{scrubbing results, spectrum before and after.}
%     \label{fig:scrub_res}
%     \caption{\ronak{These plots will be replaced with results from "comparing with full Hesian".}}
% \end{figure}





%%% Older writing, to be moved/incorporated.
% \subsection{Older writing, to be moved/incorporated.}
% With the above proofs-of-concept in hand, we move to evaluating the full unlearning procedure on larger neural network models.


% We compare our L-FOCI unlearning procedure against a full Hessian step as in \cite{sekhari2021remember}, using a simple XXX model trained on CIFAR-10 such that the computation is feasible. 

% \paragraph{Implementation Details.} Our selection procedure is implemented in PyTorch, with a framework for mapping and vectorizing convolutional layers and filters selected by L-FOCI over hypercolumns. Hessians are approximated using finite-differencing, and computed using the gradient vectors at the last two epochs of training. Full experimental details along with code for our experiments can be found in the supplement.

% \begin{figure}
%     \centering
%     \includegraphics[width=0.22\textwidth]{5_unlearn/figs/spectra/cifar10_simple_before_train.pdf}
%     \includegraphics[width=0.22\textwidth]{5_unlearn/figs/spectra/cifar10_simple_after_14.pdf}
%     \caption{\ronak{TODO: update with newer version, fixed models.}}
%     \label{fig:my_label}
% \end{figure}

% \paragraph{Unlearning Setup.} Here, we compare our L-FOCI unlearning procedure against a full Hessian step. We train a ResNet-20 model on CIFAR-10. While the model is large, we are still able to compute the full Hessian update.
%  {\color{red} TODO: model, transforms, learning rate/epochs, etc. }

% \paragraph{Baseline ablations.} {\color{red} if smaller model for full Hessian approx, details here}. Using the model above, we evaluate and compare various unlearning procedures. Table~\ref{tab:scrub} shows gradient norms and validation accuracies before and after scrubbing using naive methods of simple backpropogation, full Hessian updates as in \cite{sekhari2021remember}, random selections of layers, and selections via L-FOCI.

% \paragraph{Spectral analysis.} Using \cite{yao2020pyhessian}, we can quickly approximate the second-order eigenspectrum of a model at specific points in the input space.
%In Figure~\ref{fig:eigen}, we show the spectra of the baseline model above before and after scrubbing 


% \paragraph{Convergence.} If we have time run multiple steps, show spectrum changing?



% \subsubsection{Unlearning Identities in CelebA}
% person identification, exposing AI, data use...

% exp setup: n classes, model for IDing, scrubbing a person means scrubbing all of their images, accuracy for classification on that person decreases.



% \subsubsection{Estimating Sample Influence}
% \begin{outline}[enumerate]
% \1 identifying important samples in traditional classification
%     \2 to use FSI, need to fall back to weight 2-norm
%     \2 MNIST
%     \2 Identifying OOD/Corrupted Samples
%         \3 plots of in/out distribution a la soatto
%     \2 minimal sets for accuracy
%         \3 redo soatto plot
% \end{outline}
% \input{_theory}
\section{Discussion}
We presented an efficient solution for solving common practical multi-marginal optimal transport problems.
Our construction is significantly cheaper to compute compared to similar methods,
and allows for large numbers of distributions to be matched in common DNN pipelines. 
Our implementation allows imposing fairness constraints for a variety of applications, including those with many groups, without the need for pairwise measures.
As such, subgroup fairness \citep{kearns2018preventing} is an interesting problem setting that we believe can benefit.
% generalized Earth Mover's objective in the context of a setting where we are interested in bringing a set of distributions closer together. 
% Our construction provides significant computational gains over barycentric approaches for estimating higher dimensional distances. Taking advantage of a direct linear programming formulation, we identify readily available gradients that allow for simple incorporation of the generalized EM measure in existing optimization pipelines. We present empirical evidence that our formulation is faster and valuable in some applications.
%Future work may include exploring the practical application and value of the key theoretical result from \cite{kline2019properties} regarding 
Other properties such as 
Minkowski additivity that have not been 
explicitly leveraged in our experiments may also be a worthwhile direction to explore.
\begin{figure}
    \centering
    % \includegraphics[width=0.32\textwidth]{figs/sliced/demd_mulidim_proj_convergence.png}
    % \includegraphics[width=0.32\textwidth]{figs/sliced/groups_vs_distance.png}
    \includegraphics[width=0.4\textwidth]{6_demd/figs/sliced/demd_multidim_proj_convergence_CelebA.png}
    \caption{Sliced DEMD as a function of number of projections.}
    \label{fig:sliced}
\end{figure}
%based on specific use cases.

One limitation of DEMD
is its inability to be directly applied to distributions over multi-dimensional discrete spaces, such as latent spaces common in generative models. 
% The nature of the cost and algorithm suggest this might be a nontrivial, but potentially interesting problem to study.
% In evaluating what might enable these extensions (over images, higher-order tensors, etc.), one important piece is generalizing the Monge cost in some reasonable form, given by an ordering over the multi-dimensional space. Importantly, \textit{a Euclidean embedding of our discrete space does not result in a Monge cost. \ronak{this is true for euc dist, but squared euc dist does satisfy monge}} A sufficient ordering could be defined by a space-filling curve in continuous settings (Hilbert/Peano curves may suffice), but this would not handle arbitrary dimensional spaces unless reduced to the 1-D setting as described above (and adds incremental utility for much more complexity). 
Slicing is a heuristic that has been shown to work well. To evaluate feasibility, we embed distributions over multi-dimensional continuous spaces, take random projections over 1-D spaces, and recompute our DEMD measure. Our gains in the many-distribution setting extend here as well: over a 64-dimensional latent space embedding of CelebA, we can efficiently compute our DEMD measure over all 40 attribute subgroups, and observe convergent behavior w.r.t. the number of projections.% (Fig.~\ref{fig:sliced}).
%Our scaling above enables distance computations in this setting as well, See Appendix~\ref{sec:app-multi} for additional details.

%{\bf Ethical Considerations.} Machine learning (ML) models that have skewed performance on subsets of individuals are increasingly of concern within the community. Our proposed DEMD can be used to address general fairness problems in ML, particularly where group disparities are to be mitigated. 
%We aim to add another tool in the fairness toolbox available to ML practitioners, alongside important cultural procedures promoting equitable use, from data sources to model outputs and human audits.

% While the goal of our tool is to mitigate group disparities in practice, it is not a catch-all solution to fairness problems in machine learning. Although the results presented are promising, any application of DEMD in practice should, as always, be done with care, including both algorithmic procedures such as full hyperparameter optimization and cross validation, as well as independent, human audits of all models and results.

% An intriguing direction for future work could exploit an unexplored feature of the generalized EM functional. In~\cite{kline2019properties}, it is shown that the optimal objective value of the generalized EM program is {\em Minkowski additive} in the program data. This is a highly unusual property. It may be possible to exploit this property to other use cases in machine learning models, as well as other theoretical investigations.


% \textbf{Minkowski Additivity}
% Means we can also separate and combine functionals on different sets, and optimizing one set until it is a subset of another allows us to take advantage of the additivity.

% \textbf{Fairness Intrinsic Dimensionality}
% The support outlined above is a measure of fairness dimensionality.
% A completely fair algorithm would have a dimensionality of 0.



% \begin{ack}
% Use unnumbered first level headings for the acknowledgments. All acknowledgments
% go at the end of the paper before the list of references. Moreover, you are required to declare
% funding (financial activities supporting the submitted work) and competing interests (related financial activities outside the submitted work).
% More information about this disclosure can be found at: \url{https://neurips.cc/Conferences/2022/PaperInformation/FundingDisclosure}.


% Do {\bf not} include this section in the anonymized submission, only in the final paper. You can use the \texttt{ack} environment provided in the style file to autmoatically hide this section in the anonymized submission.
% \end{ack}

{\small
\bibliography{refs.bib}
\bibliographystyle{iclr2023_conference}
}

\newpage
% \todo{appendix if needed}
% \appendix
\section{Appendix}
\section{Theoretical Results}
\subsection{Proof of Lemma 1}
Let us take $D$ to be the training set; w.l.o.g. $z$ is the point being removed. Let the residual dataset be $D' = D \setminus {z}$.
Denote $w^-_{Full}$ as the weight parameters after doing a full Hessian update and $w^-_{Foci}$ as the weight parameters after doing a FOCI selected Hessian update.
In an ideal case, we want $(w^-_{Foci}, D')$/$(w^-_{Full}, D')$ to be as close as possible to $(w^*, D')$. Note that we consider both $(w^*, D)$ and $(w^*, D')$ to be $0$ as we don't expect model parameters to change drastically for one sample once trained to convergence. 

\begin{lemma}
The gap between the gradient residual norm of the FOCI Unlearning update in Algorithm~\ref{alg:blockunlearn} 1 and a full unlearning update via Eq.~\eqref{eq:sekhariunlearn} in the main thesis,
\begin{align}
||\nabla \mathcal{L}(w^-_{Foci},D')||_2 - ||\nabla \mathcal{L}(w^-_{Full},D')||_2
\end{align}
shrinks as $O(1/n^2)$.
\end{lemma}
\begin{proof}
Let $w$ to be a network of many linear layers with possible activation functions; we can think of the norm as the sum of norm of gradients for each layer. Hence, for any model parameters $w$ and dataset $D$, we have:
% strong assumption here, tbd add details
\begin{align}
    \label{eq:res_norm}
    ||\nabla \mathcal{L}(w,D)||_2 \coloneqq \sum_{l \in L} ||\nabla \mathcal{L}(w_l, D) ||_2
\end{align}

FOCI identifies a subset $T \subset L$ slices or layers that are to be updated. 
Let $R = L \setminus T$ be the remainder of the network which is not updated.
Hence, \ref{eq:res_norm} for $(w^-_{Foci}, D')$ can be written as:
\begin{align}
    \label{eq:foci_res_norm}
    & ||\nabla \mathcal{L}(w^-_{Foci},D')||_2 \coloneqq \sum_{l \in L} ||\nabla \mathcal{L}(w^-_{Foci_l}, D') ||_2 \\
    & = \sum_{l \in T} ||\nabla \mathcal{L}(w^-_{Foci_l}, D') ||_2 + 
     \sum_{l \in R} ||\nabla \mathcal{L}(w^-_{Foci_l}, D') ||_2 \\
    & = \sum_{l \in T} ||\nabla \mathcal{L}(w^-_{Foci_l}, D') ||_2 + 
    \sum_{l \in R} ||\nabla \mathcal{L}(w^*_{l}, D') ||_2
\end{align}
The last line follows from the fact that layers in $R$ are not updated.

We will next show how for the remainder of the dataset $D'$, the changes in $T$ propagate minimally when there are a large number of data points, $n$ in the training set.

W.L.O.G. assume that we have a $3$ layer network with the form:
\begin{align}
    (L_3(L_2(L_1(x)))
\end{align}
For the point being removed $z := (x,y)$; let $L_2$ be the intermediate layer which is selected for update by FOCI.
Before the update, activations out of $L_2$ are of the form $a_2 = L_2(L_1(x)) = L_2(a_1)$.
After the update, activations out of $L_2$ can be written as:
\begin{align}
    a_2' = L_2'(L_1(x)) &= L_2'(a_1) \\
    & = w_2' a_1 \\
    & = (w_2 + \delta_{w_2}) a_1 \\
    & = w_2 a_1 + \delta_{w_2} a_1 \\
    & = a_2 + \delta_{w_2} a_1
\end{align}
The Second line follows because $L_1$ isn't updated.
For the following layer $L_3$, we have $a_3 = L_3(a_2)$ before the update. After,
\begin{align}
    a_3' &= L_3(a_2') \\
    &= L_3(a_2 + \delta_{w_2} a_1) \\
    &= L_3(a_2) + \nabla L_3 (a_2) \delta_{w_2} a_1 + \mathcal{O}((\delta_{w_2} a_1)^2) \\
    &= L_3(a_2) + 0 + \mathcal{O}((\delta_{w_2} a_1)^2) 
\end{align}
The first-order term goes to zero, as $L_3$ has not been updated and we assume full model convergence.

For the \cite{sekhari2021remember} update.
\begin{align}
    \delta_{w_2} = \frac{1}{(n-1)}(\hat{H}^{-1})\sum_{z \in \{(x_k, y_k)\}} \nabla f(\hat{w}, z)
\end{align}
Hence, $\delta_{w_2}^2 \propto \frac{1}{n^2}$. Therefore, for large values of $n$, the third term in the equation above approaches $0$. So, $a_3' = L_3(a_2)$. This shows that propagation is minimal. Similar arguments regarding null space for over-parameterized deep networks have been mentioned in \cite{golatkar2020forgetting}. 

Now, looking back at the residual gradient norm, we have:
\begin{align}
    \label{eq:foci_res_norm_simplified}
    ||\nabla \mathcal{L}(w^-_{Foci},D')||_2 &= \sum_{l \in T} ||\nabla \mathcal{L}(w^-_{Foci_l}, D') ||_2 + \\
    & \sum_{l \in R} ||\nabla \mathcal{L}(w^*_{l}, D') ||_2
\end{align}
Based on the above argument of minimal propagation, the second term above goes to $0$ for layers/slices in $R$.
Therefore,
\begin{align}
    \label{eq:foci_res}
    ||\nabla \mathcal{L}(w^-_{Foci},D')||_2 = \sum_{l \in T} ||\nabla \mathcal{L}(w^-_{l}, D') ||_2 
\end{align}
and as such the gap between this and the full update is only the difference on the set $R$, shrinking as $O(1/n^2)$.
% \begin{align}
%     \label{eq:full_res}
%     ||\nabla \mathcal{L}(w^-_{Full},D')||_2 = \sum_{l \in L} ||\nabla \mathcal{L}(w^-_{l}, D') ||_2 
% \end{align}
% Since, FOCI selects sufficient sets, we have $T \subset L$. Hence, 
% \begin{align}
%     ||\nabla \mathcal{L}(w^-_{Foci},D')||_2 < ||\nabla \mathcal{L}(w^-_{Full},D')||_2 
% \end{align} 
\end{proof}

\subsection{Proof of Theorem~\ref{thm:fociconv}}
% Sampling with weights proportional to the Lipschitz constant of individual filters/layers is an established approach in optimization \cite{gorbunov2020unified}. We argue that L-CODEC computes an approximation to optimal sampling probabilities for updating purposes. Under a mild assumption that the sampling probabilities have \emph{full} support, it turns out that correctness of our approximate (layer/filter selection) procedure can be guaranteed for unlearning purposes using recently developed optimization tools \cite{gower2019sgd}. By adapting results from \cite{gorbunov2020unified}, we can show the following theorem that summarizes the main result of our slice-based unlearning procedure.
\begin{theorem}
Assume that layer-wise sampling probabilities are nonzero. Given (user specified) unlearning parameters $\epsilon,\delta$, the unlearning procedure in Algorithm~\ref{alg:blockunlearn} is $(\epsilon',\delta')-$forgetting where $\epsilon'>\epsilon,\delta'>\delta$ represent an arbitrary  precision (hyperparameter) required for unlearning. Moreover, iteratively applying our algorithm converges exponentially fast (in expectation) with respect to the precision gap, that is, takes (at most) $O(\log\frac{1}{\mathbf{g_{\epsilon}}}\log\frac{1}{\mathbf{g_{\delta}}})$ iterations to output such a solution where  $\mathbf{g_{\epsilon}} = \epsilon'-\epsilon>0,\mathbf{g_{\delta}}=\delta'-\delta>0$ are gap parameters.
\end{theorem}
\begin{proof}
Our proof strategy is to show that our update step in Algorithm 1 is a specific form of Randomized Block Coordinate Descent (R-BCD) method. Then, we simply apply existing convergence rates of RBCD for general smooth minimization problems.  In particular, our method can be seen as an extension of SEGA method in Corallary A.7. \cite{gorbunov2020unified} where the descent direction is provided by using inexact inverse hessian metric \cite{loizou2020convergence}.  The key difference in our setup is that the sampling probabilities are computed using the CODEC procedure instead of the random sampling at each step. We make the following three observation in our setup  that immediately asserts correctness of the procedure. 

First, by our construction in equation~\eqref{eq:perturbsupp} in the main thesis, the sampling probabilities have full support. That is, the probability of selecting a particular weight in the neural network is strictly positive since $\xi\sim\mathcal{N}(0,\sigma^2), \sigma>0$ is a continuous distribution which has unbounded support. Second, the overall rate of speed of convergence depends on the condition number of the (fixed) Hessian at the optimal solution since exact $(\epsilon,\delta)$ unlearning is equivalent to linear least squares problem. Third, our update step is equivalent to a projected (or sketched) primal step, see equation 13 in (ArXiv Version \cite{loizou2019convergenceArxiv}). From these observations, we can see that our overall method is equivalent to SEGA in \cite{gorbunov2020unified} or its noisy extension since we use only a small set of samples (to be unlearned) at each iteration. Consequently, we obtain the deterministic geometric rate of convergence (in expectation)  by applying Corollary A.8. where $\sigma$ in their work corresponds to the $\epsilon'-\epsilon>0$ gap in our setup. Now, to get the probabilistic $\epsilon',\delta'$ unlearning guarantee for the solution presented by our algorithm, we use Lemma 10 in \cite{sekhari2021remember} on the solution returned, completing our proof.
\end{proof}
% Our result is  slightly different from Nesterov's acceleration since we do not use previous iterates in a momentum or ODE like fashion. The acceleration that we obtain here is closer to primal-dual algorithms where knowing nonzero coordinates at the dual optimal solution   (better duality gap)  can simply be used to accelerate primal convergence \cite{diakonikolas2019approximate}. Moreover, since our approach is {\em randomized}, the dynamics can be better modeled using the SDE framework for unlearning purposes \cite{simsekli2020fractional}. 
% In our implementation, we do not  compute anything extra, although remains feasible for future extensions.
% And relate to additive error on $(\epsilon,\delta)$ forgetting in the beginning...
\section{Differentiable Histogramming}\label{sec:supphist}
\iffalse
\begin{algorithm}
\caption{Differentiable Histograms}\label{alg:diffhist}
\begin{algorithmic}[1] \footnotesize
\Function {Init}{$n$} {\em \hspace*{\fill} // bins, (discretization)}
    \State $r := 1/n$ {\em \hspace*{\fill} // bin size}
    \State $locs := arange(0, 1, r)$ {\em \hspace*{\fill} // bin boundaries}
\EndFunction
\Function {Forward}{$acts$}
    \State $cdfs = \sigma(acts)$ {\em \hspace*{\fill} // compute CDFs}
    \State $counts = []$
    \For{loc in locs}
        \State $dist = \lvert cdfs - loc \rvert$ {\em \hspace*{\fill} // dist. to boundary}
%        \State {\em // soft bucket count}
        \State $ct = \underset{i\in[nbins]}{\sum} \operatorname{ReLU}(r - dist[i])$ {\em \hspace*{\fill} // soft bucket count}
        \State $counts.append(ct)$
    \EndFor
    \State $out = stack(counts)$
    \State $out = out/sum(out)$
    \State \Return $out$
\EndFunction
\end{algorithmic}
\end{algorithm}
\fi

While gradients are now readily available, typical ML pipelines do not have distributions or histograms predefined at outputs which can be fed directly into 
our EMD loss. Applying existing binning procedures over the batch to estimate histograms will break the ability to autodifferentiate: soft thresholds are necessary at bin boundaries such that samples within a bin may move smoothly as needed. Algorithm \ref{alg:diffhist} provides a differentiable histogram implementation. Using a rectified linear relaxation allows for samples to have a continuous gradient towards neighboring bins.
\section{Experimental Details}
 Results reported in tables in the main paper are of the form $M\scriptscriptstyle{(SD)}$, where $M$ is the mean and $SD$ is the standard deviation calculated over replications.

{\bf Setup details.} Experiments were conducted using NumPy and PyTorch on a Intel(R) Xeon(R) CPU E5-2620 v3 @ 2.40GHz with an Nvidia Titan Xp GPU. Particular parameter settings and experimental runs can be found below or in the scripts included with the code provided in the supplement.

% {\bf Practical considerations.}
% In our case, it often happens during training that the optimal solutions may, through updates by stepping in the direction of the gradient, acquire entries that are negative. This violates an assumption that entries of the program input must be nonnegative. However, Theorem 2.2 item 1 in~\cite{kline2019properties} shows that $\phi$ possesses a kind of translation invariance. We can leverage this property to ensure that, in the event that a point escapes the nonnegative orthant of $\RR^{n}$, we can add a constant vector to the current iterate so that it again lies in the nonnegative orthant, and we can do this without changing the objective value. 
% Further, since we wish to compare probability distributions normalized so that $e'p_j=1$ for all $j$, if at some point during training a step modifies $p_j$ so that $e'p_j\not=1$, we can normalize by the positive scalar, $e'p_j>0$. In practice, since each step is a function of a relatively small learning rate, by the continuity of $\phi$, the normalization is a small perturbation of the original point. 
\section{d-Dimensional Earth Mover's Distance Background and Algorithm}\label{sec:suppdemd}
Algorithm~\ref{alg:primaldual} describes the greedy algorithm that solves both primal and dual generalized Earth mover's programs, also see \citep{kline2019properties}. The algorithm accepts $d$ distributions (i.e., histograms) $p_1,\ldots,p_d\in\RR^{n}_+$ with $e'p_j=1$ for all $j\in[d]$. 
Although the algorithm states that all histograms have the same number of bins, the algorithm can be 
easily adapted to accept as inputs $p_i\in\RR^{n_i}_+$ with $n_i\not=n_j$.
\iffalse
\begin{algorithm}
\caption{EMD Primal/Dual Algorithm}\label{alg:primaldual}
\begin{algorithmic}
\State {\bf input} $p_j\in\RR^{n}_+$ with $e'p_j=1$ , $(\forall j\in[d])$
% \State {\bf initialize}
\State {\em //iteration index, active indices, primal variables, dual variables}
\State  $k \coloneqq 0\in\ZZ$, $I \coloneqq 0\in\ZZ^{d}$, $x \coloneqq 0\in\RR^{n^{d}}$, $z_j\coloneqq 0\in\RR^{n}$, $(\forall j\in [d])$
% \State  $I \coloneqq 0\in\ZZ^{d}${\em \hspace*{\fill} // track the active indices}
% \State  $x \coloneqq 0\in\RR^{n^{d}}${\em \hspace*{\fill}  // variable of (\ref{eq:gemd})}
% \State  $z_j\coloneqq 0\in\RR^{n}$, $(\forall j\in [d])$ {\em \hspace*{\fill}  // variable of (\ref{eq:dualgeneralemd})}
\While{$I(j) \leq n$, $(\forall j\in[d])$}
    \State $s_k\coloneqq \min_{j\in[d]} p_j(I(j))$ {\em\hspace*{\fill}  // the mass to move}
    \State $x(I)\gets s_k$  {\em \hspace*{\fill} // update the EMD solution}
    \State $p_j(I(j))\gets p_j(I(j))-s_k$, $(\forall j\in d)${\em\hspace*{\fill} // shrink the data}
    \State $j^*\gets\arg\min_{j\in[d]} p_j(I(j))$
    \State $I(j^*)\gets I(j^*)+1$
    \State $k\gets k+1$
    \State $t_k\gets c(I)${\em\hspace*{\fill} // cost of this step}
    \If{ $I(j^*)\leq n$}
        \State  $z_{j^*}(I(j^*))\gets t_k-t_{k-1} + z_{j^*}(I(j^*)-1)$ {\em\hspace*{\fill} // update the dual solution}
    \EndIf
\EndWhile
\State \Return $x$, $(z_1,\ldots,z_d)$, and the objective value $\sum_k s_k t_k$.
\end{algorithmic}
\end{algorithm}
\fi
\begin{algorithm}
	\caption{ \label{alg:primaldual} EMD Primal/Dual Algorithm}
	\SetAlgoLined
	\DontPrintSemicolon
	\KwIn{$p_j\in\RR^{n}_+$ with $e'p_j=1$ , $(\forall j\in[d])$}
	\SetKwFunction{FVar}{DEMD}
	\SetKwProg{Fn}{Function}{:}{}
	\Fn{\FVar{$n$}}{
		\While{$I(j) \leq n$, $(\forall j\in[d])$} {
			$s_k\coloneqq \min_{j\in[d]} p_j(I(j))$ {\em\hspace*{\fill}  // the mass to move} \\
			$x(I)\gets s_k$  {\em \hspace*{\fill} // update the EMD solution} \\
			$p_j(I(j))\gets p_j(I(j))-s_k$, $(\forall j\in d)${\em\hspace*{\fill} // shrink the data} \\
			$j^*\gets\arg\min_{j\in[d]} p_j(I(j))$ \\
			$I(j^*)\gets I(j^*)+1$ \\
			$k\gets k+1$ \\
			$t_k\gets c(I)${\em\hspace*{\fill} // cost of this step} \\
			\If{$I(j^*)\leq n$} {
				$z_{j^*}(I(j^*))\gets t_k-t_{k-1} + z_{j^*}(I(j^*)-1)$ {\em\hspace*{\fill} // update the dual solution}
			}
		}
		\KwRet{$x$, $(z_1,\ldots,z_d)$, and the objective value $\sum_k s_k t_k$}
	}
\end{algorithm}
The algorithm has explicit terminal conditions for the main while loop. In the worst case the number of iterations can be $nd$.
% \subsection{Multidimensional Extensions via Slicing}

Here we demonstrate the ability to drop-in replace existing schemes for computing Wasserstein-style distances when the distributions of interest exist in $\RR^p, p > 1$. Although this manuscript focuses on analysis in the base case, empirically we find that $d$-EMD is able to outperform existing sinkhorn methods for multiple distributions in higher dimensions. Particularly, when the number of distributions is high and the discretization level or number of bins is reasonable, sliced sinkhorn barycenter approaches fail miserably with any practical setting of the sinkhorn regularizer weight. The parameter-free $d$-EMD is able to more consistently estimate the high-dimensional distance.
\section{A Note on Extended Ethics Discussion}

As discussed in the main paper discussion,
the primary application of our proposed construction is to reduce invariance over a particular set of features. In practice, with respect to typical machine learning models and pipelines, 
this corresponds to minimizing performance difference as measured across subgroups within the data corresponding to a minority or protected subsets of samples or individuals.
While the construction can be applied to 
any existing ML pipeline,
we do not claim to provide a catch-all solution for group disparity that may be inherent to the data or exacerbated by the choice of the ML model that is being used.
As always, care needs to be taken when working with sensitive data or models which may have disparate impacts on different groups.
We point interested readers to the following extensive surveys and references therein regarding various methods and procedures for addressing and dealing with bias and unfairness in ML problems, and the potential danger associated with using models without care: \citep{mehrabi2021survey,leavy2018gender,o2016weapons,d2017conscientious,rakova2021responsible}.

We also note that the final multi-marginal GAN image translation application could be used to generate so-called ``deepfakes." 
While the results of our algorithm are comparable to existing works, we believe that existing methods of identifying deepfakes would work well, and that the methods provided here and in the original  paper \cite{cao2019multi} would require significant effort to be made practical for much larger scale images.


% \input{__notes}
% \input{__todos_log}

\end{document}
